\subsection{Navigating \AFiii with inter-domain distance restraints}
\label{subsec:guided-diffusion}

Next, we describe a method for imposing a restraint on the generative process of \AFiii so that the CVs approaches a user-specified target values. 
Specifically, while \AFiii employs a score-based diffusion model, we compute the likelihood of the generated structure with respect to the specified CVs and add to the original \AFiii score a driving force that increases this likelihood. With this modified score, the generated structures becomes explicitly conditioned on the CVs.

Let $a$ denote an amino acid sequence and $X=(x_1,\ldots,x_m)$ the 3D coordinates of all atoms, where $x_i\in\mathbb{R}^3$.
We write $X_t$ for the state at diffusion time $t\in[0,1]$.
Sampling the generative process of \AFiii can be written as the reverse-time variance-preserving SDE (VP-SDE):
\begin{equation}
    dX_t = -\Bigl(\tfrac{1}{2}X_t + \nabla_{X_t} \log \pt{X_t}\Bigr) \beta_t dt + \sqrt{\beta_t} d\bar{W}_t,
    \label{eq:af3-sde}
\end{equation}
where $\beta_t$ is the noise schedule, $\bar{W}_t$ is a standard Wiener process, and the score $\nabla_{X_t}\log \pt{X_t}$ is approximated by a denoising network $s_\theta(X_t,t,a)$. We integrate \Cref{eq:af3-sde} backward from $t=1$ to $t=0$ in the generative process.

Let $\phi_{\text{object}} \in \mathbb{R}^D$ denote the differences in the CVs between the generated structure and the target structure, where $D$ is the dimension of the CVs, and let $\phi: X \to \mathbb{R}^D$ be a differentiable mapping from structures to this feature space.
By Bayes’ rule, the conditional score decomposes as
\begin{equation}
    \nabla_{X_t}\log p_t \bigl(X_t \big| \phi_{\text{object}}, a \bigr)
    = \nabla_{X_t} \log \pt{X_t} + \nabla_{X_t} \log \ptc{\phi_{\text{object}}}{X_t}.
    \label{eq:score-decomp}
\end{equation}
Accordingly, we bias the reverse-time dynamics by adding the guidance term:
\begin{equation}
    dX_t = -\Bigl(\tfrac{1}{2}X_t + \nabla_{X_t}\log \pt{X_t} + \eta_t \nabla_{X_t}\log \ptc{\phi_{\text{object}}}{X_t}\Bigr)\,\beta_t\,dt + \sqrt{\beta_t}\,d\bar{W}_t,
    \label{eq:af3-guided}
\end{equation}
so that the process is guided toward $\phi(X_0) \approx \phi_{\text{object}}$.
Here, $\eta_t$ is a time-dependent guidance strength.

In practice, we replace the guidance score by the negative gradient of a squared-error loss:
\begin{equation}
    \mathcal{L}_{\text{MSE}}(X) \;=\; \bigl\|\phi(X)-\phi_{\text{object}}\bigr\|_2^2,
\end{equation}
\begin{equation}
    \nabla_{X_t} \log \ptc{\phi_{\text{object}}}{X_t} \approx - c_t \nabla_{X_t} \mathcal{L}_{\text{MSE}}(X_t),
\end{equation}
which is consistent with an isotropic Gaussian model $p_t \bigl( \phi_{\text{object}} | X_t,a \bigr) \propto \exp \big( - \| \phi(X_t) - \phi_{\text{object}} \|_2^2 / ( 2 \sigma_t^2 ) \big)$, in which case $c_t = 1/(2\sigma_t^2)$. 
The scale $c_t$ can be absorbed into $\eta_t$, which is analyzed in \Cref{app:guidance_sch}.

