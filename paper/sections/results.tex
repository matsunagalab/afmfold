\section{Results}
\label{sec:results}

\subsection{Verification using pseudo-AFM images: adenylate kinase}
\label{subsec:ak_results}

adenylate kinase (\AK) is a well-studied monomeric enzyme known for its large-scale conformational transitions. It is composed of three relatively rigid domains: the central CORE domain (residues 1–29, 68–117, and 161–214), the AMP-binding domain (AMPbd; residues 30–67), and the lid-like ATP-binding domain (LID; residues 118–167). Experimental and computational studies have suggested that upon ligand binding, the enzyme undergoes a transition from an inactive open conformation to an active closed conformation (see \Cref{fig:overview}~(c)). Here, we evaluated how accurately \Model reconstructs 3D conformations of \AK from artificially created pseudo ``experimental'' AFM images. 

\paragraph{Quality of the conformational ensemble generated by \AFiii}
First, we checked the quality of the conformational ensemble generated by \AFiii used for the training of the g-CNN. Here, the conformations were generated by navigating \AFiii with CV values on a grid as described in \Cref{subsubsec:afm-prep}. As the CVs, we used the inter-domain distances of all possible combinations of domain pairs (LID--CORE, CORE--AMPbd, and AMPbd--LID). To obtain physically plausible structures as accurately as possible via navigating to the target CVs, we systematically evaluated generated structures using various navigating schedules with different strengths and onset timings, and selected an optimal schedule strategy (see SI section S1 for details). \Cref{fig:ak_cnn_training}~(b) shows the generated conformations projected to the LID--CORE, and CORE--AMPbd distances (colored by the AMPbd--LID distance), in comparison with \Cref{fig:ak_cnn_training}~(a), which shows the projection of the conformations sampled by a 450~ns MD simulation. Compared with the MD simulation, the conformations generated by \AFiii adequately cover a broad conformational space of \AK,  including both closed (PDB ID: 1AKE) and open (PDB ID: 4AKE) crystal structures. This means that the g-CNN trained on this ensemble can learn the broad relationship between AFM images and the inter-domain distances of \AK.

\paragraph{Accuracy of collective variables estimated by the g-CNN}
Next, we investigated whether the trained g-CNN accurately estimates CV values from given pseudo-AFM images. Here, we estimated CV values from pseudo-AFM images emulated using randomly chosen conformations from the MD simulation trajectory. The pseudo-AFM images were generated with the settings described in \Cref{tab:afm_settings}. \Cref{fig:ak_cnn_training}~(c) shows the distribution of the root squared error in the CV space between the estimated values with the trained g-CNN and the ground-truth values. With the exceptions of some outlier errors greater than 0.500~nm, the trained g-CNN achieves the mean root squared error of 0.254~nm, which is enough resolution to capture intermediate conformational states of \AK. 

\paragraph{Accuracy of conformations estimated by \Model}
Then, we investigated the accuracy of the estimated conformations of our \Model framework. As a demonstration, we first estimated the closed and open conformations of \AK from the pseudo-AFM images emulated from the closed and open crystal structures (the images are shown in \Cref{fig:overview}~(b)), respectively. \Cref{fig:overview}~(c) shows the estimated structures by the protocol of \Model (i.e., g-CNN and \AFiii navigation). The RMSD between the ground-truth and reconstructed conformations was 0.216~nm for the closed form and 0.176~nm for the open form, respectively. Considering that the RMSD between the closed and open crystal structures is 0.715~nm, \Model estimation is accurate enough to identify the two states. The c.c. between the AFM images created from the estimated conformations and the given ``experimental'' pseudo-AFM image is 0.997 for the closed form and 0.998 for the open form, respectively (here, \Cref{eq:cc} was used to compute the c.c.). 

We then evaluated the RMSDs of the conformations estimated from the pseudo-AFM images emulated from randomly chosen conformations from the MD simulation trajectory. \Cref{fig:ak_cnn_training}~(e) shows the distribution of the RMSDs between the ground-truth and estimated conformations. Overall, the RMSD values are low and, aside from a small number of outliers, most RMSDs are distributed around the mean value of approximately 0.281~nm. \Cref{fig:ak_cnn_training}~(d) shows the scatter plot of the RMSD and the root squared errors of the predicted CVs between the ground-truth and estimated conformations. Although a reasonable correlation is observed between the two measures, some scatter was observed in RMSD at similar error levels; that is, even with comparable CV errors, the RMSDs can vary. Upon close inspection of the generated structures, we found that this arises because a given set of current CVs (inter-domain distances) does not always uniquely determine the domain arrangement. Specifically, either hinge-bending (open--close transition) or shear-like (twisted) domain motions can yield similar inter-domain distances, despite resulting in structurally distinct conformations. This degeneracy in the CV--structure mapping explains the occurrence of RMSD outliers even when the predicted CV values are accurate.

%A second factor that degrades reconstruction accuracy is the quality of the images themselves.
%\Cref{fig:ak_accuracy}~(a) shows the distribution of RMSD between the ground-truth and predicted conformations.
%While most cases exhibit low RMSD, a subset shows high RMSD in \Cref{fig:ak_accuracy}~(a).
%\Cref{fig:ak_accuracy}~(b) shows the distribution of the c.c. between the reference images and the %pseudo-AFM images generated by rigid-body fitting of the predictions to the reference.
%In contrast to RMSD, \Cref{fig:ak_accuracy}~(b) is skewed toward high c.c.
%Indeed, \Cref{fig:ak_accuracy}~(c) indicates that even when RMSD is relatively high, c.c. can remain high (upper-right region), suggesting that some of the reference AFM image does not sufficiently determine inter-domain distances.
%Taken together, despite the presence of systematic errors stemming from methodological limitations of \Model, the framework reliably distinguishes large-scale domain motions with high probability overall.

\begin{figure}[H]
    \centering
    \includegraphics[width=1.0\linewidth]{images/figures-02.png}
    \caption{
        \textbf{Training data and estimation accuracy of \Model}.
        \textbf{(a)} Projection of training data into the inter-domain distance space. The two cross marks represent the open (4AKE) and closed (1AKE) crystal structures from PDB, respectively.
        \textbf{(b)} Projection of a 450 ns MD simulation trajectory into the inter-domain distance space.
        \textbf{(c)} Distribution of the root squared error in the inter-domain distance space between the g-CNN's estimations and the ground-truth. The inter-domain distance was estimated by the trained g-CNN from pseudo-AFM images, which were generated from randomly selected molecular dynamics simulation conformations. The average of the root squared errors is indicated by the broken line (0.254 nm). 
        \textbf{(d)} Scatter plot of the root squared error in the inter-domain distance space and the heavy-atom RMSDs from the ground-truth conformation. 
        The heavy-atom RMSD was computed between the conformation generated with \AFiii navigating using the predicted CV values, and the ground-truth conformation.
        \textbf{(e)} Density of the heavy-atom RMSDs. The average of the RMSD is indicated by the broken line (0.281 nm).
        }
    \label{fig:ak_cnn_training}
\end{figure}

\paragraph{Correlation coefficient and robustness against noise}
%So far, we have evaluated the estimation accuracy of \Model using noise-free pseudo-AFM images. To assess whether our approach is applicable to real experimental data, we next investigated the prediction accuracy using pseudo-AFM images with added noise.
%From \cref{fig:noise_robustness}, we observed the following. 
To investigate the applicability of our approach to real experimental data, we next focused on two aspects: (i) the development of evaluation metrics that can be computed without ground-truth structures and (ii) the assessment of robustness in structure estimation accuracy against noise.
As for the first aspect, following the previous study \cite{Niina2020}, we propose to use the c.c. between the given experimental image and a pseudo-AFM image created by rigid-body fitting of the estimated structure to the experimental image, maximizing the c.c. values.
\Cref{fig:noise_robustness}~(a) shows the density of the c.c. between the ``experimental'' images (pseudo-AFM images created from randomly selected structures from the MD data) and pseudo-AFM images created by rigid-body fitting of the estimated structures. With the exceptions of some outlier values, the distribution is centered around the average value of 0.998, indicating that the estimated structures are in good agreement with the given images. \Cref{fig:noise_robustness}~(b) shows the scatter plot of the c.c. and the RMSD. 
A clear negative correlation is observed between the c.c. and the RMSD, suggesting that, even when ground-truth structures are unavailable and RMSD cannot be computed, as is often the case with real experimental data, the comparison of c.c.\ values provides a useful indication of the reliability of the estimated structures.

\begin{figure}[htbp]
    \centering
    \includegraphics[width=\linewidth]{images/figures-03.png}
    \caption{
        \textbf{Correlation coefficients between the given ``experimental'' images and the pseudo-AFM images generated from the estimated structures, and robustness against noise}. 
        Panels \textbf{(a)} and \textbf{(b)} present the evaluations under noise-free conditions, whereas panels \textbf{(c--f)} evaluate the robustness of the estimations against noise.
        \textbf{(a)} Distribution of correlation coefficients (c.c.) between the given images (``experimental'' pseudo-AFM images) and pseudo-AFM images, each of which was generated by a rigid-body fitting of an estimated conformation to the reference image, maximizing the c.c. value. The average of c.c. is indicated by the broken line (0.998). 
        \textbf{(b)} Scatter plot of the c.c. and the RMSD. 
        %indicating a negative correlation between image similarity and structural accuracy.
        \textbf{(c)} Scatter plot of the c.c. and the RMSD using ``experimental'' pseudo-AFM images contaminated by Gaussian noise with a standard deviation of 0.3 nm. 
        %Compared with panel (b), which was computed using noise-free images, the c.c. values are overall lower, and this is accompanied by a general upward shift in the RMSD distribution.
        \textbf{(d)} Examples of ``experimental'' pseudo-AFM images used for estimating structures at different noise levels in the investigation of noise robustness. 
        The images were generated from randomly selected structures from the molecular dynamics simulation trajectory and then adding Gaussian noise with various standard deviations.
        \textbf{(e)} Heat map of the mean RMSD between the estimated structures and the ground-truth structures 
        for various noise levels in the training and ``experimental'' pseudo-AFM data. 
        %under conditions where noise was introduced in both the training and reference data. 
        %The x axis indicates the noise level in the training data, and the y axis indicates that in the reference data. 
        Each mean RMSD was computed for 20 independent structure estimations. 
        \textbf{(f)} Heat map of the mean c.c. between the rigid-body fitted pseudo-AFM images and the ``experimental'' pseudo-AFM images under the same noise levels as in panel (e). 
    }
    \label{fig:noise_robustness}
\end{figure}

Next, we focused on the second aspect, the assessment of robustness in structure estimation accuracy against noise. So far, we have used noise-free pseudo-AFM images for both the training and inference stages. To assess the robustness of the estimation against noise, we first added noise to ``experimental'' pseudo-AFM data in the inference stage using Gaussian noise with a standard deviation of 0.3 nm (which is a typical noise level for HS-AFM images \cite{Niina2020}). 
\Cref{fig:noise_robustness}~(c) shows the scatter plot of the c.c. and the RMSD using the noisy ``experimental'' pseudo-AFM images. Still, a clear negative correlation is observed between the c.c. and the RMSD, suggesting that the c.c. values would work even when noisy real experimental data are used.

Furthermore, for systematic evaluation of the robustness against various noise levels, 
we systematically varied the noise levels added to both the training and the ``experimental'' pseudo-AFM data. \Cref{fig:noise_robustness}~(d) illustrates examples of pseudo-AFM images at different noise levels, and \Cref{fig:noise_robustness}~(e) shows a heat map of the mean RMSD between the estimated and ground-truth structures over the range of noise levels. Notably, we found that adding noise to the ``experimental'' pseudo-AFM images consistently led to substantial reductions in prediction accuracy. In contrast, adding noise to the training data alone had a relatively minor effect on the estimation accuracy; indeed, the best performance was achieved when noise-free training data were used. This indicates that, while robustness to noise in the inputs (``experimental'' images) is limited, the presence or absence of noise in the training data does not substantially affect accuracy, with noise-free training data providing the most reliable results. \Cref{fig:noise_robustness}~(f) shows a heat map of the mean c.c. between the rigid-body fitted pseudo-AFM images and the ``experimental'' pseudo-AFM images under the same noise levels as in \Cref{fig:noise_robustness}~(e).

%When we used noise-free AFM images as training data, the model accurately extracted features and inferred structures from noise-free inputs. However, it lacked robustness to noisy inputs: inference accuracy degraded substantially when the reference images (inputs) were heavily contaminated by noise. By contrast, using images with comparatively high noise levels for training introduced a trade-off between peak accuracy and robustness. Specifically, even when the input noise matched the training noise, the prediction accuracy did not reach that of the noise-free case. Nevertheless, unlike the noise-free setting, the model retained reasonable accuracy even when the input noise level differed from that of the training data. 

%In general, the noise level of the training data should match that of the target images. Based on these results, the recommended practice depends on the noise level of the reference images. If the reference images have high noise levels ($\gtrsim 0.3$ nm), one should accept somewhat lower reliability, but reasonable accuracy can still be expected even when the input noise level differs from the training data. Conversely, if the reference images are low-noise, highly reliable predictions are achievable, but only when the input noise level is similar to that of the training data; images with markedly different noise levels should not be used as references.

\subsection{Application to real experimental AFM images: a flagellar protein \flhac}
\label{subsec:flhac_results}

FlhA is a key component of the flagellar protein export system, and its C-terminal domain, \flhac, helps transport proteins by assisting their binding to the exporter. 
Structurally, it comprises four domains (\acdi, \acdii, \acdiii, and \acdiv) and a linker to the transmembrane domain (FlhA\textsubscript{TM}) (\Cref{fig:flhac_statistic}~(a)). 
Previous studies using X-ray crystallography and mutational analysis indicate that hinge motions among these domains influence transport activity and motility \cite{saijo-hamano2010flha}. 
%Here, we applied \Model to real experimental AFM images of \flhac to evaluate its ability to reconstruct the 3D structure of \flhac from AFM images. 

Here, to evaluate the applicability of \Model to real AFM data \cite{inoue_structural_2019}, we estimated the conformations of \flhac using \Model from 159 HS-AFM images of the \flhac monomer.
It should be noted that, whereas conventional flexible fitting methods for capturing structural changes typically require hours to days of computation per AFM image even with a coarse-grained model \cite{Niina2020}, \Model allows estimation of conformations from each AFM image in approximately one minute on a single GPU (considering training time, total computation time is less than the conventional flexible fitting methods). This enabled, for the first time, the processing of as many as 159 AFM images in this study.

\paragraph{Training of the g-CNN}
Conformations for training the g-CNN were generated by navigating \AFiii with CV values on a grid as described in \Cref{subsubsec:afm-prep}. As the CVs, we used the inter-domain distances of domain pairs, \acdi--\acdiv, \acdii--\acdiii, and \acdii--\acdiv. 
The generated ensembles are shown with their MolProbity scores in \Cref{fig:flhac_selection}. 
The g-CNN was trained on the training data to learn the relationship between AFM images and the CVs.
%We used interdomain distances between \acdi--\acdiv, \acdii--\acdiii, and \acdii--\acdiv as CVs for generating training data and estimating their coordinates from AFM images using the g-CNN.

\begin{figure}[htbp]
    \centering
    \includegraphics[width=0.9\linewidth]{images/figures-04.png}
    \caption{
        \textbf{\flhac structure and estimated structures compared with the crystal structure}.
        \textbf{(a)} Crystal structure of \flhac (PDB ID: 3A5I). 
        The four domains are colored as follows: dark blue, linker; sky blue, \acdi; green, \acdii; yellow, \acdiii; and orange, \acdiv. 
        \textbf{(b,e,h)} Experimental AFM images selected from real experimental data. 
        \textbf{(c,f,i)} Crystal structures fitted to the corresponding experimental AFM images through rigid-body fitting maximizing the c.c. value. Pseudo-AFM images generated from them are shown in the background.
        \textbf{(d,g,j)} Estimated structures by \Model, fitted to the corresponding experimental AFM images through rigid-body fitting maximizing the c.c. value. Pseudo-AFM images generated from them are shown in the background.
    }
    \label{fig:flhac_results}
\end{figure}

\paragraph{Evaluating estimated conformations by \Model}
\Cref{fig:flhac_results} shows the estimated structures of three AFM images selected from the total 159 AFM images. We performed rigid-body fitting of the estimated structures to the real AFM image by exhaustively rotating and translating and finding the pose that maximizes the c.c. value, and then their poses and pseudo-AFM images were drawn for comparison with the real AFM images. 
As a comparison, we performed the same fitting procedure for the crystal structure (PDB ID: 3A5I) of \flhac to the three AFM images (also shown in \Cref{fig:flhac_results}). 
%rigid-body fitting for both structures estimated by \Model and the PDB database structure.
%For all cases, \Model reproduced more sililar images with reference images than the PDB database structure (see \cref{fig:flhac_statistic} (b)).
Although it is difficult to distinguish by visual inspection alone, in all three cases, the c.c. value between the pseudo-AFM image and the real AFM image was consistently higher for the estimated structure than for the crystal structure. It should be noted, however, that the improvement in c.c. is only slight. This is because the overall c.c. value is largely dominated by whether the molecule is correctly positioned at the center of the image, while the influence of finer structural details such as subtle surface features is relatively minor in comparison, as already shown in the previous verification for \AK. 

\begin{figure}[htbp]
    \centering
    \includegraphics[width=0.9\linewidth]{images/figures-04.2.png}
    \caption{
        \textbf{Analysis of the estimated structures by \Model}.
        \textbf{(a)} 
        Comparison of correlation coefficients between pseudo-AFM images generated from the \Model estimations and the experimental AFM images (159 frames) with those of the crystal structure. 
        The $x$-axis represents the mean of $\mathrm{c.c._{AFM-Fold}}$ and $\mathrm{c.c._{Crystal}}$, 
        and the $y$-axis shows their difference ($\mathrm{c.c._{AFM-Fold}} - \mathrm{c.c._{Crystal}}$). 
        %Most data points lie above zero, indicating that \Model generally achieves higher agreement with the reference AFM images than the static PDB structure.
        \textbf{(b)} Distribution of the inter-domain distances of the \Model estimations compared with AI-based structure generation models.
        \textbf{(c)} Time series of inter-domain distances estimated by \Model for the consecutive 159 frames of the experimental AFM images.
        \textbf{(d)} 
        Auto-correlation functions of the inter-domain distance as a function of time lag. 
        The solid line shows the results from \Model estimations, while the dashed line corresponds to temporally shuffled data. 
    }
    \label{fig:flhac_statistic}
\end{figure}

Next, we evaluated the estimated conformations by \Model using the total 159 images. \Cref{fig:flhac_statistic}~(b) shows the comparison of the c.c. values between the estimated structures by \Model and the crystal structure with the real AFM images. Again, in almost all cases, the c.c. value for the estimated structure was higher than that for the crystal structure. Interestingly, some of the c.c. values of \Model take higher values when the estimated structures have more open conformations in \Cref{fig:flhac_statistic}~(b) (measured by the distance between \acdii and \acdiv, which serves as a useful indicator to distinguish between open and closed conformations). This might suggest that the flexibility of the inter-domain motions of \flhac is well captured by \Model.
%This is because the open conformation has more surface features that are more easily captured by the AFM images.

To quantify the flexibility of the inter-domain motions captured by \Model, we computed the distribution density of the inter-domain distance between \acdii and \acdiv for all the estimated structures (\Cref{fig:flhac_statistic}(c)). 
For comparison, we also generated conformations using AI-based structure generation models, AlphaFlow \cite{jing2024alphaflow}, BioEmu \cite{bioemu2025}, and \AFii with MSA subsampling \cite{delAlamo2022sampling}. 
These methods are expected to approximate long-time behaviors of \flhac that are not accessible by brute-force MD simulations. 
Interestingly, compared with these models, \Model tends to estimate more open conformational states. 
%Compared with these models, \Model tends to predict more open conformations.
There are three possible explanations for why \Model tends to estimate more open conformational states than the AI-based structure generation models. First, HS-AFM captures molecular dynamics on a timescale of 1~ms per frame, which may reflect longer-timescale motions than those considered by AlphaFlow or BioEmu, thereby enabling the observation of more open structures. Second, interactions between the molecule and the stage during AFM imaging may stabilize conformations with larger surface areas in contact with the stage, thus favoring more open structures. Third, errors in the structural estimation by \Model could result in an overrepresentation of highly open conformations with large entropy. These factors together may account for the observed tendency toward open states in our estimations.

To verify the third possibility (large estimation errors in \Model), 
we investigated the time-series of the estimated structures by \Model (\Cref{fig:flhac_statistic}~(c)). 
Since the 159 AFM images are a set of consecutive frames as time-series data, we can expect some temporal correlations in the estimated structures if the estimation errors are small. 
Among the inter-domain distances shown in \Cref{fig:flhac_statistic}~(c), the distance between \acdii and \acdiv serves as a key indicator distinguishing between open and closed conformations. Examination of the time-series reveals that there are periods where this distance is large and others where it is small, 
%indicating that the estimated structures fluctuate between more open and more closed states over time. 
Importantly, this temporal pattern suggests the presence of time correlation in the estimated structures. Despite the fact that \Model estimates each frame independently without considering temporal information, the observed time correlation in estimated inter-domain distances implies that the estimation errors are smaller than the scale of the observed fluctuations. 
%This consistency supports the reliability of our \Model estimations and suggests that the structural changes identified reflect genuine molecular motions captured in the AFM sequence, rather than just noise or random estimation error.  

The temporal correlation of the inter-domain distances was quantified by the auto-correlation function (\Cref{fig:flhac_statistic}~(d)).
The solid lines show the auto-correlations from \Model estimations, while the dashed lines show the auto-correlations from temporally shuffled surrogate data (i.e., the frame order of the inter-domain distances time-series is shuffled). 
In particular, the auto-correlation for the temporally shuffled data remains low (within ±0.2) across all lag times, indicating little to no time correlation as expected for randomly ordered data. In contrast, the auto-correlation for the \Model estimations shows a pronounced positive correlation for lag times up to 5.0 ms and exhibits larger negative values for lag times beyond 15 ms, resulting in oscillatory behavior. 
Overall, these results demonstrate that \Model is effective for capturing the molecular conformational dynamics from the time-series of HS-AFM images. 
%This pattern suggests that the estimated structures from \Model not only retain temporal continuity but may also capture underlying dynamic fluctuations in the molecular conformation.

%We analyzed the predicted conformations as a time series.
%As shown in \cref{fig:flhac_statistic}(e), the time correlation between adjacent frames (with a time lag of 1~ms) remains relatively high ($ \sim 0.5$).
%Moreover, \Model does not simply predict constant values; as illustrated in \cref{fig:flhac_statistic}(d), the predicted inter-domain distances exhibit substantial transitions over time.
%When we randomly shuffled the frame order, the temporal correlation nearly vanished, as indicated by the dotted line in \cref{fig:flhac_statistic}(e).

%Although \Model relies solely on AFM images, its ability to predict similar structures for neighboring frames suggests that it is robust to experimental noise and variations in probe shape.
%This consistency indicates that \Model can extract meaningful structural information from real AFM data.
