\section{Results}
\label{sec:results}

\subsection{Verification using pseudo-AFM images: Adenylate Kinase}
\label{subsec:ak_results}

Adenylate Kinase (\AK) is a well-studied monomeric enzyme known for its large-scale conformational transitions. It is composed of three relatively rigid domains: the central CORE domain (residues 1–29, 68–117, and 161–214), the AMP-binding domain (AMPbd; residues 30–67), and the lid-like ATP-binding domain (LID; residues 118–167). Experimental and computatinoal studies have suggested that upon ligand binding, the enzyme undergoes a transition from an inactive open conformation to an active closed conformation (see \Cref{fig:overview}~(c)). Here, we evaluated how accurately \Model reconstructs 3D conformations of \AK from AFM images.

\paragraph{Quality of the conformational ensemble generated by \AFiii}
First, we checked the quality of the conformational ensemble generated by \AFiii used for the training of the g-CNN. Here, the conformations were generated by steering \AFiii with CV values on a grid as described in \Cref{subsubsec:afm-prep}. As the CVs, we used the inter-domain distances of all possible combinations of domain pairs (LID--CORE, CORE--AMPbd, and AMPbd--LID). To obtain physically plausible structures as accurately as possible via steering to the target CVs, we systematically compared results using various steering schedules with different strengths and onset timings, and selected an optimal schedule strategy (see Supporting Information S1 for details). \Cref{fig:ak_cnn_training}~(b) shows the generated conformations projected to the LID--CORE, and CORE--AMPbd distances (colored by the AMPbd--LID distance), in comparison with \Cref{fig:ak_cnn_training}~(a), which shows the projection of the conformations sampled by a 450~ns MD simulation. Compared with the MD simulation, the conformations generated by \AFiii adequately cover a broad conformational space of \AK,  including both closed (PDB ID: 1AKE) and open (PDB ID: 4AKE) crystal structures. This means that the g-CNN trained on this ensemble can learn the broad relationship between AFM images and the inter-domain distances of \AK.

\paragraph{Accuracy of the g-CNN}
Next, we investigated whether the trained g-CNN accurately estimates CV values from given pseudo-AFM images. Here, we estimated CV values from pseudo-AFM images emulated using randomly chosen conformations from the MD simulation trajectory. The pseudo-AFM images were generated with the settings described in \Cref{tab:afm_settings}. \Cref{fig:ak_cnn_training}~(c) shows the distribution of the root squared error in the CV space between the estimated values with the trained g-CNN and the ground-truth values. With the exceptions of some outlier errors greater than XXXXX~nm, the trained g-CNN achieves the average squared error of 0.254~nm, which is enough resolution to capture intermediate conformational states of \AK. 

\paragraph{Accuracy of \AFiii steering}
Then, we investigated the accuracy of the estimated conformatinos of our \Model framework. As a demonstration, we first estimated the closed and open conformations of \AK from the pseudo-AFM images emulated from the closed and open crystal structures (the images are shown in \Cref{fig:overview}~(b)), respectively. \Cref{fig:overview}~(c) show the estimated structures by the protocol of \Model (i.e., g-CNN + \AFiii steering). The RMSD between the ground-truth and reconstructed conformations was 0.216~nm for the closed form and 0.176~nm for the open form, respectively. Considering that the RMSD between the closed and open crystal structures is 0.715~nm, \Model estimation is accurate enough to identify the two states. Using \Cref{eq:cc} to compute the c.c. between the AFM images created from the estimated conformations and the given ``experimental'' pseudo-AFM image, we obtained 0.997 for the closed form and 0.998 for the open form, respectively. 

We then evaluated the RSMDs of the conformations estimated from the pseudo-AFM emulated from radomly chosen confrmations from the MD simulation trajectory. \Cref{fig:ak_cnn_training}~(e) shows the distribution of the RMSDs between the ground-truth and estimated conformations. Overall, the RMSD values are low and, aside from a small number of outliers, most RMSDs are distributed around the mean value of approximately 0.281~nm. \Cref{fig:ak_cnn_training}~(d) shows the scatter plot of the RMSD and the root squared errors of the predicted CVs between the ground-truth and estimated conformations. Although a reasonable correlation is observed between the two measures, some scatter observed in RMSD at similar squared error levels; that is, even with comparable CV errors, the RMSDs can vary. Upon close inspection of the generated structures, we found that this arises because a given set of current CVs (inter-domain distances) does not always uniquely determine the domain arrangement. Specifically, both hinge-bending (open--close transition) and shear-like (twisted) domain motions can yield similar inter-domain distances, despite resulting in structurally distinct conformations. This degeneracy in the CV--structure mapping explains the occurrence of RMSD outliers even when the predicted CV values are accurate.

%A second factor that degrades reconstruction accuracy is the quality of the images themselves.
%\Cref{fig:ak_accuracy}~(a) shows the distribution of RMSD between the ground-truth and predicted conformations.
%While most cases exhibit low RMSD, a subset shows high RMSD in \Cref{fig:ak_accuracy}~(a).
%\Cref{fig:ak_accuracy}~(b) shows the distribution of the c.c. between the reference images and the %pseudo-AFM images generated by rigid-body fitting of the predictions to the reference.
%In contrast to RMSD, \Cref{fig:ak_accuracy}~(b) is skewed toward high c.c.
%Indeed, \Cref{fig:ak_accuracy}~(c) indicates that even when RMSD is relatively high, c.c. can remain high (upper-right region), suggesting that some of the reference AFM image does not sufficiently determine inter-domain distances.
%Taken together, despite the presence of systematic errors stemming from methodological limitations of \Model, the framework reliably distinguishes large-scale domain motions with high probability overall.

\begin{figure}[H]
    \centering
    \includegraphics[width=0.7\linewidth]{images/figures-02.png}
    \caption{
        \textbf{Training data and accuracy of the group-invariant CNN and \AFiii steering}.
        \textbf{(a)} Projection of a 450 ns MD simulation trajectory into the inter-domain distance space. 
        \textbf{(b)} Projection of \AFiii-generated training data into the inter-domain distance space. 
        \textbf{(c)} Density of root squared error in the inter-domain distance space between the trained g-CNN's estimations and the ground-truth. The inter-domain distance was estimated by the trained g-CNNfrom pseudo-AFM images, which were generated from randomly selected MD conformations. The average of the root squared errors is indicated by the broken line (0.254 nm). 
        \textbf{(d)} Scatter plot of the root squared error in the inter-domain distance space and the heavy-atom RMSDs from the ground-truth conformation. 
        The heavy-atom RMSD was computed between the conformation generated with \AFiii steering using the predicted CV values, and the ground-truth conformation.
        \textbf{(e)} Density of the heavy-atom RMSDs. The average of the RMSD is indicated by the broken line (0.281 nm).
        }
    \label{fig:ak_cnn_training}
\end{figure}

\paragraph{Correlation coefficient and robustness against noise}
%So far, we have evaluated the estimation accuracy of \Model using noise-free pseudo-AFM images. To assess whether our approach is applicable to real experimental data, we next investigated the prediction accuracy using pseudo-AFM images with added noise.
%From \cref{fig:noise_robustness}, we observed the following. 
To investigate the applicability of our approach to real experimental data, we next focused on two aspects: (i) the development of evaluation metrics that can be computed without ground-truth structures and (ii) the assessment of robustness in structure estimation accuracy against noise.
As for the first aspect, following the previous study \cite{Niina2020}, we propose to use the c.c. between the given images ("experimental" pseudo-AFM images) and images, each of which is an experimental image and a pseudo-AFM image created by rigid-body fitting of the estimated structure to the experimental image maximizing the c.c. 
\Cref{fig:noise_robustness}~(a) shows the density of the correlation coefficients (c.c.) between the ``experimental'' images (pseudo-AFM images created from randomly selected structures from the MD data) and pseudo-AFM images created by rigid-body fitting of the estimated structures. With the exceptions of some outlier values, the distribution is centered around the average value of 0.998, indicating that the estimated structures are in good agreement with the given images. \Cref{fig:noise_robustness}~(b) shows the scatter plot of the c.c. and the RMSD. 
A clear negative correlation is observed between the c.c. and the RMSD, suggesting that, even when ground-truth structures are unavailable and RMSD cannot be computed, as is often the case with real experimental data, the comparison of c.c.\ values provides a useful indication of the reliability and accuracy of the estimated structures.

Next, we focused on the second aspect, the assessment of robustness in structure estimation accuracy against noise. So far, we have used noise-free pseudo-AFM images for both the training and inference stages. To assess the robustness of the estimation against noise, we firstly added noise to ``experimental'' pseudo-AFM data in the inference stage using Gaussian noise with a standard deviation of 0.3 nm (which is a typical noise level for HS-AFM images \cite{Niina2020}). 
\Cref{fig:noise_robustness}~(c) shows the scatter plot of the c.c. and the RMSD using the noisy ``experimental'' pseudo-AFM images. Still, a clear negative correlation is observed between the c.c. and the RMSD, suggesting that the c.c. values would work even when noisy real experimental data are used.

Furthermore, for systematic evaluation of the robustness against various noise levels, 
we systematically varied the noise levels added to both the training and the ``experimental'' pseudo-AFM data. \Cref{fig:noise_robustness}~(d) illustrates examples of pseudo-AFM images at different noise levels, and \Cref{fig:noise_robustness}~(e) shows a heat map of the mean RMSD between the estimated and ground-truth structures over the range of noise levels. Notably, we found that adding noise to the ``experimental'' pseudo-AFM images consistently led to substantial reductions in prediction accuracy. In contrast, adding noise to the training data alone had a relatively minor effect on the estimation accuracy; indeed, the best performance was achieved when noise-free training data were used. This indicates that, while robustness to noise in the inputs (``experimental'' images) is limited, the presence or absence of noise in the training data does not substantially affect accuracy, with noise-free training data providing the most reliable results. \Cref{fig:noise_robustness}~(f) shows a heat map of the mean c.c. between the rigid-body fitted pseudo-AFM images and the ``experimental'' pseudo-AFM images under the same noise levels with the \Cref{fig:noise_robustness}~(e).

%When we used noise-free AFM images as training data, the model accurately extracted features and inferred structures from noise-free inputs. However, it lacked robustness to noisy inputs: inference accuracy degraded substantially when the reference images (inputs) were heavily contaminated by noise. By contrast, using images with comparatively high noise levels for training introduced a trade-off between peak accuracy and robustness. Specifically, even when the input noise matched the training noise, the prediction accuracy did not reach that of the noise-free case. Nevertheless, unlike the noise-free setting, the model retained reasonable accuracy even when the input noise level differed from that of the training data. 

%In general, the noise level of the training data should match that of the target images. Based on these results, the recommended practice depends on the noise level of the reference images. If the reference images have high noise levels ($\gtrsim 0.3$ nm), one should accept somewhat lower reliability, but reasonable accuracy can still be expected even when the input noise level differs from the training data. Conversely, if the reference images are low-noise, highly reliable predictions are achievable, but only when the input noise level is similar to that of the training data; images with markedly different noise levels should not be used as references.

\begin{figure}[htbp]
    \centering
    \includegraphics[width=\linewidth]{images/figures-03.png}
    \caption{
        \textbf{Correlation coefficient and robustness against noise}. 
        Panels (a) and (b) present the evaluations under noise-free conditions, whereas panels (c–f) evaluate the robustness of the estimations against noise.
        \textbf{(a)} Density of correlation coefficients (c.c.) between the given images (``experimental'' pseudo-AFM images) and pseudo-AFM images, each of which was generated by a rigidi-body fitting of a esimated conformation to the reference image maximizing teh c.c. The average of c.c. is indicated by the borken line (0.998). 
        \textbf{(b)} Scatter plot of the c.c. and the RMSD. 
        %indicating a negative correlation between image similarity and structural accuracy.
        \textbf{(c)} Scatter plot of the c.c. and the RMSD using ``experimental'' pseudo-AFM images contaminated by Gaussian noise with a standard deviation of 0.3 nm. 
        %Compared with panel (b), which was computed using noise-free images, the c.c. values are overall lower, and this is accompanied by a general upward shift in the RMSD distribution.
        \textbf{(d)} Examples of pseudo-AFM images at different noise levels used in the investigation of noise-robustness. 
        The images were generated from randomly selected structures from the MD data and then adding Gaussian noise with various standard deviations.
        \textbf{(e)} Heat map of mean RMSD between the estimated structures and the ground-truth structures 
        for various noise levels in the training and ``experimental'' pseudo-AFM data. 
        %under conditions where noise was introduced in both the training and reference data. 
        %The x axis indicates the noise level in the training data, and the y axis indicates that in the reference data. 
        Each mean RMSD was computed for 20 independent structure estimations. 
        \textbf{(f)} Heat map of mean c.c. between the rigid-body fitted pseudo-AFM images and the ``experimental'' pseudo-AFM images under the same noise levels with the panel (e). 
    }
    \label{fig:noise_robustness}
\end{figure}


\subsection{Application to real experimental AFM images: A Flagellar Protein \flhac}
\label{subsec:flhac_results}

\flhac is a key component of the flagellar protein export system, and its C-terminal domain, \flhac, helps transport proteins by assisting their binding to the exporter.
Structurally, it comprises four domains (\acdi, \acdii, \acdiii, and \acdiv) and a linker to the transmembrane domain (FlhA\textsubscript{TM}). See \Cref{fig:flhac_statistic}~(a) for details.
Previous studies using X-ray crystallography and mutational analysis indicate that hinge motions among these domains influence transport activity and motility \cite{saijo-hamano2010flha}.

\paragraph{Preprocessing}
We selected \acdi--\acdiv and \acdii--\acdiii as inter-domain distance features.
The candidate conformations and their MolProbity scores are shown in \Cref{fig:flhac_selection}.

\begin{figure}[htbp]
    \centering
    \includegraphics[width=0.9\linewidth]{images/figures-04.png}
    \caption{
        \textbf{Results for \flhac}.
        \textbf{(a)} Reference AFM images selected from real experimental data. 
        \textbf{(b)} Results of rigid-body fitting. 
        For each reference image, we reconstructed 3D conformations and database structures, applied rigid-body fitting, and plotted the resulting optimal poses along with their corresponding pseudo-AFM images. 
        For each row, the top figures show the results for \Model’s predictions, and the bottom figures show the results for the database structures. 
        \textbf{(c)} Visualization of the rigid-body fitting process. 
        For all sampled SO(3) rotations, the corresponding correlation coefficients (c.c.) are color-mapped: blue indicates lower c.c. (worse fits), while red indicates higher c.c. (better fits). 
        The star denotes the optimal pose. 
        \textbf{(d)} Histogram of c.c. values. 
        We uniformly sampled SO(3) rotations and computed the c.c. for each case. 
        The histogram shows the resulting distribution, and the top-right annotation reports both the maximum c.c. (i.e., at the optimal pose) and the mean c.c. value.
    }
    \label{fig:flhac_results}
\end{figure}

\paragraph{Predictions from real AFM images}
To evaluate the applicability of \Model to real AFM data, we selected three snapshots from HS-AFM movies of the \flhac monomer and predicted a conformation for each.
The first two frames (upper and middle panels in \Cref{fig:flhac_results}~(a)) were visually chosen because their overall size and shape resembled the X-ray crystal structures obtained from the PDB.
The final frame (lower panel in \Cref{fig:flhac_results}~(a)) was selected because it appeared to have a considerably larger size than the crystal structure.

For each image, we predicted the conformation and searched for the pose that achieved the highest c.c. through rigid-body fitting.
For comparison, rigid-body fitting was also performed using crystal structures (in each case, the top row in (b,c,d) corresponds to \Model predictions, and the bottom row to the crystal structures).

For all cases, the rigid-body fitting was run with sufficiently long searches to ensure convergence.
\Cref{fig:flhac_results}~(c) shows all c.c. values sampled during the fitting process, demonstrating that the rotational space was thoroughly explored and that the global maximum was likely identified.

As a result, \Model reproduced every reference image more accurately than crystal structures.
\Cref{fig:flhac_results}~(d) shows histograms of the c.c. values sampled during rigid-body fitting.
Looking at the highest c.c. in each case (denoted as “Max” in the figure), we found that the best c.c. values were consistently higher for \Model's prediction than for the crystal structures.
The pseudo-AFM images corresponding to these maxima are shown in \Cref{fig:flhac_results}~(b).

%Interestingly, there were no obvious shape differences visible to the human eye between the pseudo-AFM images generated from \Model predictions and the crystal structure.
%Indeed, when visually comparing the upper and lower results in \cref{fig:flhac_results}~(b), it is difficult for human eyes to judge which is better, yet \Model successfully distinguishes subtle image features to predict the correct conformation.

Interestingly, we found that \Model tends not to fit the images too tightly, maintaining a reasonable level of generalization.
A particularly notable case is the final image.
For this elongated structure, \cite{Niina2020} reported an extended structure, including partial unfolding.
In contrast, \Model predicted a conformation that remains relatively close to the native structure, involving only minor structural changes.
This difference can be attributed to the exclusion of physically unrealistic conformations during the preparation of CNN training data, 
suggesting that \Model tends to generate predictions within a physically reasonable conformational space.

%\clearpage

\paragraph{Statistical assessment}
To further observe the statistical performance of \Model, we carried out conformation predictions using many experimental AFM images.
Although images with obvious noise were excluded from the analysis, 
in total, conformations were predicted for 160 AFM frames, most of which were consecutive snapshots (corresponging to about 13 seconds).

For each reference image, we performed rigid-body fitting in the same way as described above, using both the predicted conformations and the crystal structures for comparison.
\Cref{fig:flhac_statistic}~(b) summarizes the results of these fittings.
\Model reproduced almost all reference images more accurately than the crystal structures.

In \Cref{fig:flhac_statistic}~(b), the color of each point indicates the distance between \acdii and \acdiv of the \Model's predictions.
Smaller distances correspond to closed conformations, while larger distances correspond to open conformations (see \Cref{fig:flhac_statistic}~(a)).
Interestingly, the c.c. values do not show a clear dependence on this distance.
This means that \Model can flexibly predict either open or closed states depending on the input image, while still achieving high agreement with the AFM data.

Also, it is interesting to note that, the inter-domain distance in the crystal structure is 3.71 nm.
Even when the predicted conformations had similar distances, \Model still produced higher c.c. values with the experimental images.
This suggests that \Model adjusts additional degrees of freedom beyond the \acdii–\acdiv motion to better match the observed images.
In the current setup, the CNN predicts two CVs: the distances between \acdi–\acdiv and between \acdii–\acdiii.
%These results imply that increasing the number of CVs predicted by the CNN could further improve structural accuracy.

\Cref{fig:flhac_statistic}~(c) shows the distribution of the \acdii–\acdiv distances predicted by \Model.
The gray dashed line marks the corresponding distance in the crystal structure.
Compared with distributions sampled in previous studies, this predicted distribution is slightly broader, covering both open and closed conformations.

Even though the distribution is not thermodynamically reliable, it is important that the distribution still spans the relevant conformational space.
This indicates that reweighting the ensemble with an appropriate potential could help filter out unreliable predictions.
Ideally, conformational prediction itself should be drived by equilibrium distribution, 
allowing conformations to be generated according to their stability.
We leave this for future work and will discuss it further in the \cref{sec:discussion}.

\begin{figure}[htbp]
    \centering
    \includegraphics[width=1.0\linewidth]{images/figures-05.png}
    \caption{
        \textbf{Statistical results for \flhac}.
        \textbf{(a)} Crystal structure of \flhac (PDB ID: 3A5I). The four domains are colored as follows: dark blue, linker; sky blue, \acdi; green, \acdii; yellow, \acdiii; and orange, \acdiv.
        \textbf{(b)} Comparison of cross-correlation (c.c.) values between \Model predictions and the crystal structure for various reference images. The color represents the distance between \acdii and \acdiv.
        \textbf{(c)} Distribution of the \acdii–\acdiv distances suggested by the structures predicted by \Model. The distribution was obtained from dynamics corresponding to approximately 13 seconds of observation.
    }
    \label{fig:flhac_statistic}
\end{figure}

