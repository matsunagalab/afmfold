\section{Results}
\label{sec:results}

\subsection{Adenylate Kinase (\AK)}
\label{subsec:ak_results}

%To first validate the effectiveness of \Model, we employed 
\AK is a well-studied monomeric enzyme known for its large-scale conformational transitions. 
%, catalyzing the reversible phosphoryl transfer reaction (ATP + AMP $\leftrightarrow$ 2ADP) that regulates cellular energy homeostasis. 
\AK is composed of three relatively rigid domains: the central CORE domain (residues 1–29, 68–117, and 161–214), the AMP-binding domain (AMPbd; residues 30–67), and the lid-like ATP-binding domain (LID; residues 118–167). 
Experimental and computatinoal studies have suggested that upon ligand binding, the enzyme undergoes a transition from an inactive open conformation to an active closed conformation (see \Cref{fig:overview}~(c)). 
Here, we evaluated how accurately \Model predicts 3D conformations of \AK from AFM images.

%\paragraph{CNN training.}

First, we checked the quality of the conformations generated by \AFiii for training data of the g-CNN. The candidate conformations were generated with \AFiii  (by steering \AFiii with CV values on a grid as described in \Cref{subsubsec:afm-prep}). 
%are projected them to the inter-domain distance space (\Cref{fig:ak_cnn_training}~(b)).
%Results are shown in \cref{fig:ak_cnn_training}~(b). 
Here, we used the inter-domain distances of all possible combinations of domain pairs (LID--CORE, CORE--AMPbd, and AMPbd--LID) as CVs. 
% is equivalent to $\R^3$.
\Cref{fig:ak_cnn_training}~(b) shows the generated conformations projected to the LID--CORE, and CORE--AMPbd distances (colored by the AMPbd--LID distance). 
Also, \Cref{fig:ak_cnn_training}~(a) shows the projection of the conformations sampled by a 450 ns MD simulation (\Cref{fig:ak_cnn_training}~(a)), which covers both closed (PDB ID: 1AKE) and open (PDB ID: 4AKE) crystal structures.
Comparison of the two projections shows that the generated conformations adequately covered a broad conformational space of \AK.
%\Cref{fig:ak_cnn_training}~(a) is that of representive conformations in a 450 ns MD trajectory. 
%In this simulation, multiple transitions between the open and closed conformations were observed, suggesting that a wide conformational space was explored. 
%Importantly, the sampling broadly covered the conformational space that was explored by MD. 
%even though we used no prior knowledge about MD simulations.

%\paragraph{Steering ability.}
Next, we investigated how accurately \AFiii steering follows the specified CVs (inter-domain distances).
Since the CV values in a grid used for generating the candidate conformations include unphysical values, which may not be physically achievable, we evaluated the accuracy of steering by using the CV values of the crystal structures. % and the conformations sampled by the MD simulation. 
%\AFiii generated an open conformation when we did not impose any restraints. 
%As an initial test of \Model's ability to steer \AFiii's sampling trajectory, we generated pseudo-AFM images from the closed (PDB ID: 1AKE) and open (PDB ID: 4AKE) crystal structures, and reconstructed 3D conformations from each image. 
\Cref{fig:overview}~(b) and (c) show the generated structues by applying the CV values of the closed and open crystal structures.  
%The results are shown in \Cref{fig:overview}~(b), (c).
The RMSD between the ground-truth and reconstructed conformations was 0.216 nm for the closed form and 0.176 nm for the open form, respectively. 
Considering that the RMSD between the closed and open crystal structures is 0.715 nm, the steering was accurate enough to distinguish the two states. 
The pseudo-AFM images generated from predicted conformations after rigid-body fitting of the generated structures are displayed in \Cref{fig:overview}~(b). 
Using \Cref{eq:cc} to compute the c.c., we obtained 0.997 for the closed form and 0.998 for the open form.

\begin{figure}[H]
    \centering
    \includegraphics[width=0.7\linewidth]{images/figures-02.png}
    \caption{
        \textbf{Training data and prediction accuracy of the group-invariant CNN}.
        \textbf{(a)} Projection of a 450 ns MD simulation trajectory into the inter-domain distance space. 
        \textbf{(b)} Projection of the candidate conformations into the inter-domain distance space. 
        \textbf{(c)} A histogram of root squared error in the inter-domain distance space between the pre-trained CNN's outputs and ground-truth. 
        The CNN predicted inter-domain distance from pseudo-AFM images, which ware generated from randomly selected MD conformations. 
        The root mean squared error (RMSE) was 0.183 [nm].
        \textbf{(d)} A scatter plot of root squared error in the inter-domain distance space and ${\mathrm{C}\alpha}$-atom RMSDs. 
        The RMSD was computed between the conformation generated with \AFiii steering using the predicted CV values, and the ground-truth conformation.
        }
    \label{fig:ak_cnn_training}
\end{figure}

%\paragraph{Accuracy evaluation.}
Finally, we investigated the accuracy of the estimated conformatinos of our \Model framework. Here, we estimated CV values from pseudo-AFM images generated from representative conformations of the 450 ns MD simulation, and steered \AFiii using the estimated CV values to generated 3D conformations. The pseudo-AFM images were generated with the same settings as those in \Cref{tab:afm_settings}. 
%To investigate the prediction accuracy of \Model, we extracted representative conformations from the 450 ns MD simulation and generated pseudo-AFM images (settings identical to those in \cref{tab:afm_settings}).
%Using these images, we predicted 3D conformations with the \Model framework.
\Cref{fig:ak_cnn_training}~(c) shows the histogram of the root squared error (RMSE) in the CV space between the predicted values with the trained g-CNN and the ground-truth values.
% and \cref{fig:ak_accuracy}~(a)--(c).
%An informative observation is that we can analyze sources of error in \Model's 3D conformation reconstruction.
%From the histogram of the root square error between the pre-trained CNN predictions and the ground-truth inter-domain distances (\cref{fig:ak_cnn_training}~(c))
With the exceptions of some outlier errors greater than RMSE of 0.5 nm, the trained g-CNN achieves the average RMSE of 0.183 nm, which is enough resolution to capture intermediate conformational states. 

\Cref{fig:ak_cnn_training}~(d) shows the ${\mathrm{C}\alpha}$-atom RMSDs between the generated conformations with \AFiii steering and the ground-truth conformations.
Based on a linear regression, the CNN’s typical prediction error corresponds to an RMSD of 0.245 nm. 
However, \Cref{fig:ak_cnn_training}~(d) also shows cases where the RMSD remains relatively large even when the CNN predictions are accurate: specifically, points in the upper-left region exhibit small RMSE but large RMSD.
Close inspection of such cases reveals that this discrepancy arises when domains move along a shear direction: inter-domain distances remain similar, but other degrees of freedom differ, yielding a larger RMSD.

A second factor that degrades reconstruction accuracy is the quality of the images themselves.
\Cref{fig:ak_accuracy}~(a) shows the distribution of RMSD between the ground-truth and predicted conformations.
While most cases exhibit low RMSD, a subset shows high RMSD in \Cref{fig:ak_accuracy}~(a).
\Cref{fig:ak_accuracy}~(b) shows the distribution of the c.c. between the reference images and the pseudo-AFM images generated by rigid-body fitting of the predictions to the reference.
In contrast to RMSD, \Cref{fig:ak_accuracy}~(b) is skewed toward high c.c.
Indeed, \Cref{fig:ak_accuracy}~(c) indicates that even when RMSD is relatively high, c.c. can remain high (upper-right region), suggesting that some of the reference AFM image does not sufficiently determine inter-domain distances.
Taken together, despite the presence of systematic errors stemming from methodological limitations of \Model, the framework reliably distinguishes large-scale domain motions with high probability overall.

\begin{figure}[htbp]
    \centering
    \includegraphics[width=\linewidth]{images/figures-03.png}
    \caption{
        \textbf{Evaluation of prediction accuracy}. 
        \textbf{(a)} Histogram of RMSD between predicted and ground-truth structures (mean: 0.273 nm). 
        \textbf{(b)} Histogram of correlation coefficients (C.C.) between reference images and rigid-body fitted predicted images (mean: 0.997). 
        \textbf{(c)} Scatter plot of C.C. and RMSD, indicating a negative correlation between image similarity and structural accuracy.
    }
    \label{fig:ak_accuracy}
\end{figure}

\paragraph{Noise Robustness}
To assess how \Model performs under varying noise conditions, we evaluated its robustness to image noise.
From \cref{fig:noise_robustness}, we observed the following.
When we used noise-free AFM images as training data, the model accurately extracted features and inferred structures from noise-free inputs.
However, it lacked robustness to noisy inputs: inference accuracy degraded substantially when the reference images (inputs) were heavily contaminated by noise.
By contrast, using images with comparatively high noise levels for training introduced a trade-off between peak accuracy and robustness.
Specifically, even when the input noise matched the training noise, the prediction accuracy did not reach that of the noise-free case.
Nevertheless, unlike the noise-free setting, the model retained reasonable accuracy even when the input noise level differed from that of the training data.

In general, the noise level of the training data should match that of the target images.
Based on these results, the recommended practice depends on the noise level of the reference images.
If the reference images have high noise levels ($\gtrsim 0.3$ nm), one should accept somewhat lower reliability, but reasonable accuracy can still be expected even when the input noise level differs from the training data.
Conversely, if the reference images are low-noise, highly reliable predictions are achievable, but only when the input noise level is similar to that of the training data; images with markedly different noise levels should not be used as references.

\subsection{\flhac}
\label{subsec:flhac_results}

FlhA is a key component of the flagellar protein export system, and its C-terminal domain, \flhac, helps transport proteins by assisting their binding to the exporter.
Structurally, \flhac comprises four domains (\acdi, \acdii, \acdiii, and \acdiv) and a linker to the transmembrane domain (FlhA\textsubscript{TM}). See \cref{fig:flhac_statistic}~(a) for details.
Previous studies using X-ray crystallography and mutational analysis indicate that hinge motions among these domains influence transport activity and motility \cite{saijo-hamano2010flha}.

\paragraph{Preprocessing}
We selected \acdi--\acdiv and \acdii--\acdiii as inter-domain distance features.
The candidate conformations and their MolProbity scores are shown in \cref{fig:flhac_selection}.

\begin{figure}[htbp]
    \centering
    \includegraphics[width=0.9\linewidth]{images/figures-04.png}
    \caption{
        \textbf{Results for \flhac}.
        \textbf{(a)} Reference AFM images selected from real experimental data. 
        \textbf{(b)} Results of rigid-body fitting. 
        For each reference image, we reconstructed 3D conformations and database structures, applied rigid-body fitting, and plotted the resulting optimal poses along with their corresponding pseudo-AFM images. 
        For each row, the top figures show the results for \Model’s predictions, and the bottom figures show the results for the database structures. 
        \textbf{(c)} Visualization of the rigid-body fitting process. 
        For all sampled SO(3) rotations, the corresponding correlation coefficients (c.c.) are color-mapped: blue indicates lower c.c. (worse fits), while red indicates higher c.c. (better fits). 
        The star denotes the optimal pose. 
        \textbf{(d)} Histogram of c.c. values. 
        We uniformly sampled SO(3) rotations and computed the c.c. for each case. 
        The histogram shows the resulting distribution, and the top-right annotation reports both the maximum c.c. (i.e., at the optimal pose) and the mean c.c. value.
    }
    \label{fig:flhac_results}
\end{figure}

\paragraph{Predictions from real AFM images}
To evaluate the applicability of \Model to real AFM data, we selected three snapshots from HS-AFM movies of the \flhac monomer and predicted a conformation for each.
The first two frames (upper and middle panels in \cref{fig:flhac_results}~(a)) were visually chosen because their overall size and shape resembled the X-ray crystal structures obtained from the PDB.
The final frame (lower panel in \cref{fig:flhac_results}~(a)) was selected because it appeared to have a considerably larger size than the crystal structure.

For each image, we predicted the conformation and searched for the pose that achieved the highest c.c. through rigid-body fitting.
For comparison, rigid-body fitting was also performed using crystal structures (in each case, the top row in (b,c,d) corresponds to \Model predictions, and the bottom row to the crystal structures).

For all cases, the rigid-body fitting was run with sufficiently long searches to ensure convergence.
\Cref{fig:flhac_results}~(c) shows all c.c. values sampled during the fitting process, demonstrating that the rotational space was thoroughly explored and that the global maximum was likely identified.

As a result, \Model reproduced every reference image more accurately than crystal structures.
\Cref{fig:flhac_results}~(d) shows histograms of the c.c. values sampled during rigid-body fitting.
Looking at the highest c.c. in each case (denoted as “Max” in the figure), we found that the best c.c. values were consistently higher for \Model's prediction than for the crystal structures.
The pseudo-AFM images corresponding to these maxima are shown in \cref{fig:flhac_results}~(b).

%Interestingly, there were no obvious shape differences visible to the human eye between the pseudo-AFM images generated from \Model predictions and the crystal structure.
%Indeed, when visually comparing the upper and lower results in \cref{fig:flhac_results}~(b), it is difficult for human eyes to judge which is better, yet \Model successfully distinguishes subtle image features to predict the correct conformation.

Interestingly, we found that \Model tends not to fit the images too tightly, maintaining a reasonable level of generalization.
A particularly notable case is the final image.
For this elongated structure, \cite{Niina2020} reported an extended structure, including partial unfolding.
In contrast, \Model predicted a conformation that remains relatively close to the native structure, involving only minor structural changes.
This difference can be attributed to the exclusion of physically unrealistic conformations during the preparation of CNN training data, 
suggesting that \Model tends to generate predictions within a physically reasonable conformational space.

\clearpage

\paragraph{Statistical assessment}
To further observe the statistical performance of \Model, we carried out conformation predictions using many experimental AFM images.
Although images with obvious noise were excluded from the analysis, 
in total, conformations were predicted for 160 AFM frames, most of which were consecutive snapshots (corresponging to about 13 seconds).

For each reference image, we performed rigid-body fitting in the same way as described above, using both the predicted conformations and the crystal structures for comparison.
\Cref{fig:flhac_statistic}~(b) summarizes the results of these fittings.
\Model reproduced almost all reference images more accurately than the crystal structures.

In \cref{fig:flhac_statistic}~(b), the color of each point indicates the distance between \acdii and \acdiv of the \Model's predictions.
Smaller distances correspond to closed conformations, while larger distances correspond to open conformations (see \cref{fig:flhac_statistic}~(a)).
Interestingly, the c.c. values do not show a clear dependence on this distance.
This means that \Model can flexibly predict either open or closed states depending on the input image, while still achieving high agreement with the AFM data.

Also, it is interesting to note that, the inter-domain distance in the crystal structure is 3.71 nm.
Even when the predicted conformations had similar distances, \Model still produced higher c.c. values with the experimental images.
This suggests that \Model adjusts additional degrees of freedom beyond the \acdii–\acdiv motion to better match the observed images.
In the current setup, the CNN predicts two CVs: the distances between \acdi–\acdiv and between \acdii–\acdiii.
%These results imply that increasing the number of CVs predicted by the CNN could further improve structural accuracy.

\Cref{fig:flhac_statistic}~(c) shows the distribution of the \acdii–\acdiv distances predicted by \Model.
The gray dashed line marks the corresponding distance in the crystal structure.
Compared with distributions sampled in previous studies, this predicted distribution is slightly broader, covering both open and closed conformations.

Even though the distribution is not thermodynamically reliable, it is important that the distribution still spans the relevant conformational space.
This indicates that reweighting the ensemble with an appropriate potential could help filter out unreliable predictions.
Ideally, conformational prediction itself should be drived by equilibrium distribution, 
allowing conformations to be generated according to their stability.
We leave this for future work and will discuss it further in the \cref{sec:discussion}.

\begin{figure}[htbp]
    \centering
    \includegraphics[width=1.0\linewidth]{images/figures-09.png}
    \caption{
        \textbf{Statistical results for \flhac}.
        \textbf{(a)} Crystal structure of \flhac (PDB ID: 3A5I). The four domains are colored as follows: dark blue, linker; sky blue, \acdi; green, \acdii; yellow, \acdiii; and orange, \acdiv.
        \textbf{(b)} Comparison of cross-correlation (c.c.) values between \Model predictions and the crystal structure for various reference images. The color represents the distance between \acdii and \acdiv.
        \textbf{(c)} Distribution of the \acdii–\acdiv distances suggested by the structures predicted by \Model. The distribution was obtained from dynamics corresponding to approximately 13 seconds of observation.
    }
    \label{fig:flhac_statistic}
\end{figure}

