\section{Related Work}
\label{sec:related}

To predict 3D structures directly from AFM images, approaches such as flexible fitting \cite{Niina2020} and NMFF-AFM \cite{Wu2024}, search for conformations such that reproduced images achieve a high correlation coefficient (CC) with the reference image. While effective, these methods depend on long simulation trajectories and thus suffer from high computational cost. 
% note: Overfitting の問題にも言及すべき?

On the other hand, generative modeling techniques inspired by nonequilibrium physics---most notably diffusion models and flow models---enable us to efficiently sample molecular conformations that follow the complex, multimodal distributions. For example, recent structure-generation foundation models, including \AFiii and \Boltzi, employ diffusion-based formulations. In particular, \Boltzi introduced \emph{boltz-steering}, a framework that imposes a customized potential to steer the sampling towards physically plausible conformations. Nevertheless, these models largely follow a ``one sequence--one structure'' paradigm, which makes it difficult to produce diverse conformational ensembles for a single amino-acid sequence. 

Several approaches have been proposed to address this problem. For example, the MSA subsampling \cite{delAlamo2022sampling} and its extensions \cite{Wayment-Steele2023MSAClustering} increase the diversity of multiple sequence alignments (MSAs) by masking parts of them. While these approaches can improve diversity in certain systems, the relationship between perturbations in the MSA and the resulting structural diversity remains unclear, and it is difficult to deliberately target specific conformations or unstable intermediate states. Moreover, the observed structural diversity is not guaranteed to follow the thermodynamic stability dictated by the real Boltzmann distribution.

Another powerful direction is the use of ensemble generative models. An early contribution in this area is the Boltzmann Generator, which was designed to learn and reproduce the Boltzmann distribution of small, specific peptides. More recently, models have been developed to generate diverse structural ensembles from sequences by learning from extensive MD simulation data across a wide range of proteins. However, obtaining sufficiently long MD trajectories for training requires substantial computational resources, and such models inevitably inherit the force-field biases present in the training data, making it challenging to surpass MD itself in terms of physical accuracy.

To mitigate these issues, recent studies have explored conditioning ensemble generative models on experimental measurements or physical prior knowledge. Examples include methods that generate NMR-consistent structural ensembles (e.g., ExDIFF and EGDIFF), approaches that design materials with specified band gaps (e.g., DiG), and models that explicitly incorporate MD potentials. Following this direction, \Model enables AFM-conditioned conformation generation, steering \AFiii's prior distribution toward a more physically realistic AFM distribution.