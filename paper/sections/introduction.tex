\section{Introduction}
\label{sec:intro}

High-speed atomic force microscopy (HS-AFM) enables direct visualization of biomolecular dynamics in solution at near-physiological conditions, providing unprecedented insights into the relationship between biomolecular conformational dynamics and its biological function \cite{ando_high-speed_2008, ando_high-speed_2022}. However, AFM measurements are inherently limited to surface topography, and the achievable spatial resolution is fundamentally constrained by the finite size of the cantilever tip \cite{villarrubia_algorithms_1997, matsunaga2023endtoend}. To address these limitations, there is a compelling need for computational modeling approaches that can integrate experimental AFM data with atomic-resolution three-dimensional structural models.

One approach is a rigid-body fitting, in which template structures are selected from structural databases or molecular dynamics simulations and then exhaustively rotated and translated to find the best pose matching the AFM image \cite{ando_high-speed_2005, scheuring_structure_2005, scheuring_high-resolution_2007,asakawa_submolecular-scale_2011, trinh_computational_2012, chaves_conformational_2013, amyot_bioafmviewer_2020, niina_rigid-body_2021}. Although powerful, this approach cannot account for conformational flexibilities or conformational changes (when the template structures in the database do not contain conformations that match the AFM image), rigid-body fitting fundamentally fails to provide accurate structural models, as it is inherently limited to the pre-existing conformational states in the template library. In real biological systems, the function of biomolecules are often coupled with the conformational flexibilities or dynamics. Thus, methods accounting for such conformational flexibilities are required for analyzing and interpreting the experimental AFM data.

In recent years, methods have been proposed that utilize molecular dynamics (MD) or Monte Carlo (MC) simulations to generate conformations consistent with AFM images. For example, in the flexible fitting \cite{Niina2020}, defining the agreement with an AFM image as an additional potential energy function $V_{\mathrm{AFM}}$, MD simulations are driven to minimize it. Although such a approach can be effective, it is constrained by the general limitations inherent in molecular simulations. In the flexible fitting, sampling is often trapped in local minima, making optimization slow when separated by high barriers. Furthermore, if the global minimum is desired, prohibitively long simulations must be performed to ensure adequate sampling of the conformational space. To reduce the computational cost of MD, NMFF-AFM \cite{Wu2024} use a subset of normal modes for finding conformations consistent with AFM images. In this case, however, the conformational space is fundamentally limited to capturing only harmonic fluctuations around a single reference structure, which prevents it from exploring large conformational changes that are often crucial for biological function. 

In the field of AI-based structure modeling or prediction, the development of sequence-to-structure models such as \AFii \cite{jumper2021highly, ahdritz2024openfold, mirdita2022colabfold}, \AFiii, and \Boltzi \cite{wohlwend2024boltz1} have demonstrated unprecedented precision in predicting the 3D fold of a protein. These models can be regarded as generative AI models that generate atomistic conformational ensembles from a given sequence. In principle, these generative AI models can generate not only specific peak structures present in the training data, but also diverse structural conformations. Moreover, it is known that the generation process can be navigated by applying conditioning or external forces during the generation stage. In the field of image generation, research has been conducted on navigating the generation process to bring generated images closer to specific targets.
%However, in practice, the resulting conformational distributions typically exhibit sharp peaks around specific stable states and thus fail to capture a diverse range of conformations.
Recently, guiding the generative process of diffusion-based structure prediction models, including \AFiii and \Boltzi, has emerged as an approach aimed at generating targeted conformational ensembles. For example, in the context of NMR and Cryo-EM data analysis \cite{maddipatla2025inverse}, it has been demonstrated that imposing guided restraints during the generative process enables the generation of ensembles consistent with experimental observations. Similarly, for Cryo-EM data \cite{raghu2025multiscale}, the incorporation of global or local density-map fitting as restraints has been proposed as a means of navigating the predicted conformations toward agreement with the experimental data.

In this study, we introduce \Model, which extends these ideas to AFM data analysis by employing structural restraints extracted from AFM images to guide the sampling trajectories of pre-trained protein structure generative models. Specifically, in the first step, a group-invariant CNN \cite{e2cnn} estimates low-dimensional collective  variables (CVs such as, inter-domain distances) from AFM images, and then these estimated values are used as restraints to guide the generative process, enabling the rapid sampling of 3D conformations consistent with AFM images. We validated \Model in twin experiments. First, using Adenylate kinase (\AK, PDBIDs: 1AKE \cite{muller1992adenylate}, and 4AKE \cite{muller1996adenylate}), we estimated the underlying CV values and accurately reconstructed the conformations close to the ground-truth. Second, we applied \Model to real experimental HS-AFM data from the flagellar protein \flhac (PDBID: 3A5I \cite{saijo-hamano2010flha}) and confirmed that it predicts conformations more consistent with AFM images than the corresponding PDB database structure fitted by rigid-body fitting calculations.

%The main contributions of this study are as follows.
%\begin{itemize}
%    \item We propose \Model, a generative AI-driven approach for reconstructing atomic-level protein structures from AFM images.
%    \item Using pseudo-AFM images of a model protein, we demonstrated that \Model can accurately reconstruct the ground-truth conformations.
%%    \item With experimental AFM data, we confirmed that \Model outperforms rigid-body fitting and provided concrete examples of its application.
%    \item In contrast to existing structure reconstruction methods from AFM images, which typically require hours to days, our approach enables structure prediction in under a minute while avoiding overfitting.
%    \item In the Appendix, we systematically evaluated guidance strength for the generative module and summarized effective strategies. We further proposed a method to sample structures from a broad conformational space at low computational cost by leveraging these strategies.
%\end{itemize}

A distinctive feature of our framework, \Model, is that all steps—from training data generation, model training, and inference, to structure generation—are integrated within a single unified pipeline, without relying on time-consuming processes such as molecular dynamics simulations. 
This enables rapid reconstruction of structures from AFM images while reducing computational cost, allowing the observation of structural dynamics from hundreds of consecutive HS-AFM images. \Model provides a powerful framework for observing conformational transitions and functional mechanisms underlying HS-AFM images. 

\begin{figure}[htbp]
    \centering
    \includegraphics[width=\linewidth]{images/figures-01.png}
    \caption{
        \textbf{Schematic overview of \Model and outcome}. AFM-based guidance navigates \AFiii and reproduces the closed and open states of \AK.
        \textbf{(a)} Schematic diagram of \Model. 
        The AFM image, shown at the lower left, serves as the input and is processed as follows: 
        (1) coordinates in the inter-domain distance space are predicted from the AFM image by a group-invariant CNN; 
        (2) at diffusion time $t > 0.54$, the mean squared error (MSE) is computed between the predicted coordinates and the generated conformations; 
        (3) the gradient of the MSE is used as a correction score to guide the generative process of \AFiii. 
        \textbf{(b)(c)} Results of structure prediction from pseudo-AFM images of \AK. 
        In (c), the blue structures (left: close; right: open) represent the reference ground-truth conformations, which were used to generate the two reference images shown at the bottom of (b). 
        Using these reference images, our method successfully predicted the red structures in (c). 
        After aligning the predicted conformations to the reference conformations by rigid-body fitting, the resulting pseudo-AFM images are shown at the top of (b), demonstrating good agreement with the reference images.
        }
    \label{fig:overview}
\end{figure}
