\setcounter{equation}{0}
\renewcommand{\theequation}{S\arabic{equation}}
\setcounter{figure}{0}
\renewcommand{\thefigure}{S\arabic{figure}}
\setcounter{table}{0}
\renewcommand{\thetable}{S\arabic{table}}
\setcounter{algorithm}{0}
\renewcommand{\thealgorithm}{S\arabic{algorithm}}

\section{Guidance scheduling}
\label{app:guidance_sch}

We study how the guidance schedule $\eta_t$ in \cref{eq:af3-guided} influences guided \AFiii structure generation.
Here, we index the reverse process by the reverse-time $\hat{t} = 1 - t \in [0,1]$, which runs from $0$ (start) to $1$ (end),

\begin{equation}
    d\hat{X}_{\hat{t}} = 
    \Bigl(
        \tfrac{1}{2}\hat{X}_{\hat{t}} + 
        \nabla_{\hat{X}_{\hat{t}}}\log \hat{p}_{\hat{t}}(\hat{X}_{\hat{t}}) + 
        \hat{\eta}_{\hat{t}} \nabla_{\hat{X}_{\hat{t}}}\log \hat{p}_{\hat{t}}(\phi_{\text{object}} | \hat{X}_{\hat{t}})
    \Bigr)
    \beta_{1 - \hat{t}} \; d\hat{t} + 
    \sqrt{\beta_{1-\hat{t}}} \; d\widetilde{W}_{\hat{t}},
    \label{eq:af3-guided-hatt}
\end{equation}
where $\hat{p}_{\hat{t}} = p_{1-\hat{t}}$, $\hat{\eta}_{\hat{t}} = \eta_{1 - \hat{t}}$ and $\widetilde{W}_{\hat{t}}$ denotes a standard Brownian motion in the forward direction of $\hat{t}$: $\widetilde{W}_{\hat{t}} = \bar{W}_1 - \bar{W}_{\,1-\hat{t}}$. 

Intuitively, the guidance strength $\hat{\eta}_{\hat{t}}$ controls how strongly the generation is pulled toward the target feature $\phi_{\text{object}}$.  
In terms of the isotropic Gaussian model
$p_t \big( \phi_{\text{object}} | X_t, a \big) \propto\; 
\exp \big(-\tfrac{1}{2\sigma_t^2}\|\phi(X_t)-\phi_{\text{object}}\|_2^2\big)$,
$\hat{\eta}_{\hat{t}}$ effectively determines the variance $\sigma_t^2$: larger values correspond to narrower distributions and thus stronger enforcement.  

Prior work commonly used simple schedules—--e.g., a constant value \cite{maddipatla2025inverse} or a mid-trajectory sigmoid \cite{raghu2025multiscale}—--
but, to our knowledge, there has not been a systematic quantification of how these choices affect generated structures.
Here, our goal is to quantify, across different guidance schedules, the trade-off between the effectiveness of the guidance
and the physical plausibility of the generated structures, and to propose an optimal scheduling strategy.

\subsection{Experimental settings}
\label{app:guidance_exp}

\paragraph{Schedule family and design of the sweep.}
We consider candidate guidance schedules $\hat{\eta}_{\hat{t}}$ from the sigmoidal family defined in \cref{eq:sigmoid}.
Specifically, we vary two hyperparameters that determine \emph{when} guidance turns on ($\hat{t}_{\text{start}}$) and \emph{how strong} it becomes at the end ($y_{\text{max}}$).
For the two hyperparameters, we select five candidate values each: 
$\hat{t}_{\text{start}}$ is chosen at evenly spaced points in $[0.4,0.9]$, 
and $y_{\text{max}}$ is chosen at evenly spaced points in $[0.6,1.6]$. 
By evaluating all combinations of these choices, we obtain $5\times 5 = 25$ distinct schedule settings in total. 
For each setting we generate five independent samples and report the mean values of the evaluation metrics.

\paragraph{System and targets.}
Here, we describe the choice of protein system and the collective variables (CVs) on which restraints are applied.  
The way restraints are imposed can in principle depend on factors such as protein size or the specific CVs chosen, 
but our goal is to provide a concrete reference point that can serve as a guideline when users consider how to set the guidance strength.  

Specifically, we focus on \flhac, and, consistent with the main text, we selected the domain pairs \acdi--\acdiv and \acdii--\acdiii as variables.  
As a baseline, preliminary \AFiii predictions without restraints on \flhac yield inter-domain distances of $(3.47\mathrm{nm},\,4.38\mathrm{nm})$.  
We then specify two target distance vectors $\phi_{\text{object}}$:  
(i) an in-distribution target, $\phi_{\text{object}}=(2.5\mathrm{nm},\,4.5\mathrm{nm})$, extracted from our in-house MD trajectory; and  
(ii) an out-of-distribution target, $\phi_{\text{object}}=(2.0\mathrm{nm},\,3.0\mathrm{nm})$, which we did not observe in the PDB or in the MD trajectory.

\begin{equation}
    \begin{split}
        \eta(\hat{t}) =
        \frac{y_{\text{max}}}{2}
        \frac{\| \delta \|_2}{\| g \|_2}
        \left(
            1 +
            \frac{\tanh\!\left( \alpha(\tau - \beta) / 2 \right)}
                 {\tanh\!\left( \alpha\beta / 2 \right)}
        \right),
        \qquad
        \tau = \frac{\hat{t} - \hat{t}_{\text{start}}}{1 - \hat{t}_{\text{start}}},
        \\[1.5ex]
        \delta = \nabla_{\hat{X}_{\hat{t}}}\log \hat{p}_{\hat{t}}(\hat{X}_{\hat{t}}),
        \quad
        g = \nabla_{\hat{X}_{\hat{t}}}\log \hat{p}_{\hat{t}}(\phi_{\text{object}} | \hat{X}_{\hat{t}}),
        \quad
        \alpha = 10.0,
        \quad
        \beta = 0.5.
        \label{eq:sigmoid}
    \end{split}
\end{equation}

\subsection{Results}
\label{app:guidance_results}

Across both the in-distribution and out-of-distribution targets, three clear patterns emerge.  

\begin{itemize}
    \item First, there is a dividing line in the MSE landscape, approximately connecting $(\hat{t}_{\text{start}}, y_{\text{max}})=(0.65,\,0.60)$ and $(0.73,\,1.60)$ (see \cref{fig:guidance_scheduling} (c, i)).
        On the right side of this line (i.e., with later $\hat{t}_{\text{start}}$ and smaller $y_{\text{max}}$), the MSE between the target CVs and the generated conformation's CVs rises sharply. 
    \item Second, the MolProbity-score contours are roughly circular, centered near $(\hat{t}_{\text{start}}, y_{\text{max}})=(0.40,\,1.60)$, rather than linear (see \cref{fig:guidance_scheduling} (b, h)).
        For example, the contour at 1.9 in the in-distribution case and at 2.7 in the out-of-distribution case both illustrate this pattern. 
    \item Third, combining these two observations, we find that starting the guidance early (small $\hat{t}_{\text{start}}$) and keeping the restraint weak (low $y_{\text{max}}$) enables the application of constraints 
        without substantially compromising the physical plausibility of the generated structures.
\end{itemize}

\begin{figure}[htbp]
    \centering
    \includegraphics[width=0.9\linewidth]{images/figures-06.png}
    \caption{
        \textbf{Validation results of guidance scheduling}.
        \textbf{(a)} All guidance schedules evaluated in this study. 
        The setting used in our main experiments, $\hat{t}_{\text{start}}=0.6$ and $y_{\text{max}}=0.6$, is highlighted with a thick red line. 
        \textbf{(b)--(g)} Results for the in-distribution target. 
        \textbf{(h)--(m)} Results for the out-of-distribution target. 
        \textbf{(b), (h)} Average MolProbity scores. 
        \textbf{(c), (i)} Mean squared error (MSE). 
        \textbf{(d), (j)} Mean inter-domain distance vectors (25 points, one per schedule setting) of the generated conformations, with point colors representing the MolProbity scores. 
        \textbf{(e), (f), (g), (k), (l), (m)} Evaluation metrics obtained during MolProbity scoring: clash score, Ramachandran favored rate (\%), and rotamer outlier rate (\%).
        }
    \label{fig:guidance_scheduling}
\end{figure}

\subsection{Conclusion}
Overall, we observed the expected trade-off: stronger guidance improves agreement with the target, but tends to degrade structural plausibility. 
At the same time, we obtained an additional insight: rather than applying strong restraints only in the final stages of the reverse process, it is more effective to apply weak restraints from the early stages.  
In fact, for the out-of-distribution case, we observe a plateau around $\hat{t}_{\text{start}} \approx 0.5,\, y_{\text{max}} \approx 0.6$ where the values remain stable at approximately $\text{MSE} \approx 0.7$ and MolProbity score $\approx 2.5$.  
By contrast, near $\hat{t}_{\text{start}} \approx 0.7,\, y_{\text{max}} \approx 1.6$, the region with comparable values is much narrower (see \cref{fig:guidance_scheduling} (h,i)).  
In other words, in the latter setting, even small deviations from the optimal point cause either the MSE or the MolProbity score to rise sharply.

Based on these observations, in our study we adopt $\hat{t}_{\text{start}}=0.6$ and $y_{\text{max}}=0.6$ that allow effective guidance without substantially degrading structural quality.

\section{Model Implementation Details}
\label{app:additional_alg}

\begin{algorithm}[H]
\caption{\textsc{Forward function of $(\R^2, +) \rtimes C_8$ invariant CNN}}
\label{alg:cnscnn}
\begin{algorithmic}
\State \textbf{Input:} $I \in \mathbb{R}^{B \times D \cdot N \times H \times W}$, $D=1$, $N=8$, $H=W=35$; \\
channels = [3,6,6,12,12,8], kernel\_size = [7,5,5,5,5,5], \\
$N_{\text{hidden-layers}} = 3$, $D_{\text{hidden}} = 64$, $D_{\text{out}} = N_{\text{domain-pairs}}$
\State \textbf{Output:} $\mathbf{D} \in \mathbb{R}^{B \times D_{\text{out}}}$

\State $I \gets \mathrm{R2Conv}(I;\, \text{in}=D \cdot N,\, \text{out}=\mathrm{channels}[0] \cdot N,\, k=\mathrm{kernel\_size}[0]) \to \mathrm{InnerBatchNorm} \to \mathrm{ReLU}$
\For{$i = 1 \to \mathrm{len(channels)}-1$}
    \State $I \gets \mathrm{R2Conv}(I;\, \text{in}=\mathrm{channels}[i-1]\!\cdot N,\, \text{out}=\mathrm{channels}[i]\!\cdot N,\, k=\mathrm{kernel\_size}[i])$
    \State $I \gets \mathrm{InnerBatchNorm}(I) \to \mathrm{ReLU}(I)$
    \If{$i \bmod 2 = 0$}
        \State $I \gets \mathrm{PointwiseAvgPoolAntialiased}\!\left(I;\, \sigma=0.66,\, \mathrm{stride}=
        \begin{cases}
        2 & \text{if size odd}\\
        1 & \text{if size even}
        \end{cases}\right)$
    \EndIf
\EndFor

\State $I' \gets \mathrm{GroupPooling}(I)$ \hfill $I' \in \mathbb{R}^{B \times \mathrm{channels}[-1] \times H' \times W'}$
\State $I'' \gets \mathrm{PointwiseAvgPool}(I')$ \hfill $I'' \in \mathbb{R}^{B \times \mathrm{channels}[-1] \times 1 \times 1}$
\State $\mathbf{z} \gets \mathrm{Flatten}(I'')$ \hfill $\mathbf{z} \in \mathbb{R}^{B \times \mathrm{channels}[-1]}$
\State $\mathbf{h} \gets \mathrm{Linear}(\mathbf{z};\, \text{in}=\mathrm{channels}[-1],\, \text{out}=D_{\text{hidden}}) \to \mathrm{BatchNorm1d} \to \mathrm{ELU}$ \hfill $\mathbf{h} \in \mathbb{R}^{B \times D_{\text{hidden}}}$
\For{$j = 1 \to N_{\text{hidden-layers}}-1$}
    \State $\mathbf{h} \gets \mathrm{Linear}(\mathbf{h};\, \text{in}=D_{\text{hidden}},\, \text{out}=D_{\text{hidden}}) \to \mathrm{ReLU}(\mathbf{h})$ \hfill $\mathbf{h} \in \mathbb{R}^{B \times D_{\text{hidden}}}$
\EndFor
\State $\mathbf{D} \gets \mathrm{Linear}(\mathbf{h};\, \text{in}=D_{\text{hidden}},\, \text{out}=D_{\text{out}})$ \hfill $\mathbf{D} \in \mathbb{R}^{B \times D_{\text{out}}}$
\State \Return $\mathbf{D}$
\end{algorithmic}
\end{algorithm}

\section{Additional Figures}
\label{app:additional_fig}

\begin{figure}[H]
    \centering
    \includegraphics[width=0.95\linewidth]{images/figures-05.png}
    \caption{
        \textbf{Validation results of noise robustness}.
        \textbf{(a)} AFM images at various noise levels, where the noise level denotes the standard deviation of Gaussian noise applied in pseudo-AFM images. 
        \textbf{(b)} Results of conformation reconstruction from AFM images with different noise levels, evaluated in RMSD. The horizontal axis indicates the noise level of the training images, and the vertical axis indicates the noise level of the reference images during inference. The pseudo-AFM images for training were generated from the training conformations (see \cref{subsubsec:afm-prep}), whereas the reference AFM images for validation were generated from MD simulation structures as well as in the other results. For each combination, we predicted conformations from 20 different images, and the average RMSD is plotted. 
        \textbf{(c)} Under the same conditions as in (b), the success rate of structural prediction was plotted, defining success as RMSD $< 0.3$ nm.
    }
    \label{fig:noise_robustness}
\end{figure}

\begin{figure}[H]
    \centering
    \includegraphics[width=\linewidth]{images/figures-08.png}
    \caption{
        \textbf{Selected candidate conformations for \AK and MolProbity evaluation for them}.
        \textbf{(a)} Projection of the candidate conformations into the inter-domain distance space. 
        The horizontal axis represents the LID--CORE distance, the vertical axis represents the CORE--AMPbd distance, and the color encodes the AMPbd--LID distance. 
        \textbf{(b)} MolProbity score. 
        \textbf{(c)} Clash score. 
        \textbf{(d)} Ramachandran favored rate (\%). 
        \textbf{(e)} Rotamer outlier rate (\%).
        }
    \label{fig:ak_selection}
\end{figure}

\begin{figure}[H]
    \centering
    \includegraphics[width=0.8\linewidth]{images/figures-07.png}
    \caption{
        \textbf{Selected candidate conformations for \flhac and MolProbity evaluation for them}.
        \textbf{(b)} MolProbity score. 
        \textbf{(c)} Clash score. 
        \textbf{(d)} Ramachandran favored rate (\%). 
        \textbf{(e)} Rotamer outlier rate (\%).
        }
    \label{fig:flhac_selection}
\end{figure}
