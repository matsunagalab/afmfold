\section{Methods}
\label{sec:methods}

Our \Model framework leverages \AFiii (here, we used the open-sourced Protenix \cite{bytedance2025protenix} model, a PyTorch reimplementation of \AFiii) to estimate the underlying 3D conformation from a given AFM image. To generate 3D structures consistent with AFM images using \AFiii, we follow two main steps: (i) First, we estimate low-dimensional collective variables (CVs) for the target molecule from the given AFM image using a pre-trained convolutional neural network (CNN) to predict these CVs. For CVs, we typically assume the use of inter-domain distances, especially in the case of multi-domain proteins. (ii) Next, we use the estimated CVs to add restraints to \AFiii during structure generation, producing 3D structures that are consistent with the AFM image (see \Cref{fig:overview}a).

%First, the reference AFM image is fed into a convolutional neural network (CNN), which predicts the inter-domain distances of the molecule captured in the image. 
%Then, by incorporating these predicted distances as restraints into the generation process of \AFiii, a structure consistent with the reference AFM image is obtained (see \cref{fig:overview} (a)).

In this section, we present our methodology in three parts. 
In \cref{subsec:se2cnn}, we describe our CNN architecutre to estimate the CVs from AFM images. 
In \cref{subsec:guided-diffusion}, we explain how we add restrains in the diffusion process of \AFiii using the estimated CVs. 
Finally in \cref{subsec:afm-guidance}, the overall computational protocol of our \Model framework is given, including the training of the CNN model. 

\subsection{Group-equivariant CNNs}
\label{subsec:se2cnn}

% --- Setup/context ---
%Unlike many other problems, reconstructing 3D conformations from AFM images requires verifying whether the 3D structures generated by \AFiii are consistent with a given 2D reference image. The generation process of \AFiii is guided to maximize the resulting likelihood, thereby producing conformations that agree closely with AFM images (as explained in \cref{subsec:guided-diffusion}). However, because the orientation of the molecule in the reference image is unknown and may correspond to any element of the full 3D rotation group, a naive search for the ground-truth pose is computationally prohibitive. Thus, an effective strategy is needed to assess the agreement between 2D AFM images and 3D molecular conformations. In particular, rather than transforming conformations into the AFM image format—--which is computationally demanding due to the rotational degrees of freedom—--it is more practical to derive features from AFM images that are more readily accessible from conformational side. 

In AFM 2D image analysis, similar to single particle analysis of cryoEM micrographs, the pose of the 3D structure is unknown. The alignment calculation between a 2D  image and the 3D structural pose requires exhaustive search and is computationally expensive. Moreover, if misalignment occurs, it leads to extreme deterioration in estimation accuracy. In the case of cryoEM, the influence of misalignment can be mitigated on average by a vast number of images, but in the case of AFM, since we want to perform structural estimation for each single image, misalignment cannot be tolerated. Therefore, we decided to use a rotation-equivariant CNN model (\emph{group-equivariant} CNN; g-CNN) \cite{CohenWelling2016_GCNN, CohenWelling2017_SteerableCNNs, e2cnn, WeilerEtAl2018_3DSteerableCNNs} that directly estimates structure-related CVs from 2D AFM images without performing alignment calculations. 

In steering the structure generation process of \AFiii to generate 3D structures consistent with AFM images, it would be possible to optimize by calculating image similarity each time, as done in rigid-body fitting. In this case, higher image reproduction is expected, but on the other hand, we empirically found that not only does computational cost increase, but overfitting to images causes rotamer outliers and atomic clashes, compromising the physical validity of the structure. Therefore, in this study, we decided that our g-CNN model estimates low-dimensional CVs (such as inter-domain distances in the case of multi-domain proteins) and uses them for structure generation in \AFiii.

With this motivation in mind, we summarize the theory of g-CNN (\gcnn) \cite{CohenWelling2016_GCNN, CohenWelling2017_SteerableCNNs, e2cnn, WeilerEtAl2018_3DSteerableCNNs} based approach that enables efficient comparison between AFM images and conformations. A standard CNN is naturally \emph{translation equivariant} but not rotation equivariant. In contrast, \gcnn modifies feature indexing and kernel weight sharing so that the entire network becomes equivariant with respect to a chosen rotation group. With this architecture, the same physical state yields an identical representation regardless of its placement on the image plane, thereby reducing the need for extensive data augmentation and shortening training time. Consequently, we employ the \gcnn to estimate CVs from a single AFM image.

%\paragraph{Translation equivariance}
We model an AFM image as a single-channel (scalar) field $I:\mathbb{R}^2\to\mathbb{R}$ with pixel coordinate $x=(x_1,x_2)\in\mathbb{R}^2$. 
In a standard CNN, intermediate feature maps are represented as functions $f:\mathbb{R}^2 \to \mathbb{R}^c$.
The input image is first mapped to a $c$-channel field by convolution with channel-specific filters $\{\psi_i\}_{i=1}^c$:
\begin{equation}
    (\psi * I)(x)[i] := \int_{\mathbb{R}^2} \psi_i(y)\, I(x - y)\, dy,
    \label{eq:lifting}
\end{equation}
where $*$ is convolution and $dy$ denotes the area element. 
In practice, the integral is implemented as a finite sum over pixels; we keep the continuous notation for clarity.

% --- Group and action for standard fields ---
Here, we consider the planar motion group $\R^2 \rtimes C_N$, where $C_N=\{e,r,\dots,r^{N-1}\}\subset \mathrm{SO}(2)$ is a finite rotation subgroup.
For scalar or standard $c$-channel fields (without group indexing that we later explain), 
the action $\pi(t,g)$ of translation $t\in\R^2$ and rotation $g\in C_N$ is $(\pi(t,g)u)(x)\;:=\;u\big(g^{-1}(x-t)\big)$.

Because convolution commutes with translations, a standard CNN is translation-equivariant. 
For a single convolution with a (matrix-valued) kernel $k(y)\in\R^{c_{\mathrm{out}}\times c_{\mathrm{in}}}$,
\begin{equation}
    \begin{aligned}
        \big(k * (\pi(t,e) u)\big)(x) 
        &= \int_{\mathbb{R}^2} k(y)\, u(x-y-t)\, dy \\
        &= (k * u)(x - t) = \big(\pi(t,e) (k * u)\big)(x).
    \end{aligned}
    \label{eq:trans-equiv}
\end{equation}
Since a typical activation $\sigma$ is pointwise, $\sigma$ is also translation-equivariant, $\sigma (\pi(t, e) u) = \pi(t, e) \sigma (u)$ and the property propagates through the network. By contrast, the standard convolution is generally \emph{not} rotation-equivariant.

%\paragraph{Rotation equivariance}
The key idea of \gcnns is to \emph{lift} an image to a feature field indexed by group elements and to impose weight sharing consistent with the group action.
Given a base filter $\psi:\mathbb{R}^2\to\mathbb{R}$, define its rotated copies $\psi_h(y):=\psi(h^{-1}y)$ for $h\in C_N$, and set
\begin{equation}
    v(x)[h] \;:=\; \int_{\mathbb{R}^2} \psi(h^{-1}y)\, I(x-y)\, dy,
    \qquad h\in C_N.
    \label{eq:gcnn-lifting}
\end{equation}
Thus $v:\mathbb{R}^2\times C_N\to\mathbb{R}$ is a group-indexed (orientation-channel) field.

% --- Equivariance statement (concise) ---
For such lifted fields, if the kernel satisties the following constraint about sharing weights, $k(gy)[h,h'] = k(y) \big[g^{-1}h, g^{-1}h'\big]$ for all $g, h, h' \in C_N$, then the group convolution commutes with $\pi(0,g)$:
\begin{equation}
    \big(k * (\pi(0,g)v)\big)(x)[h] \;=\; \big(\pi(0,g)(k * v)\big)(x)[h].
    \label{eq:gcnn-rot-equiv}
\end{equation}
For details, see \cite{e2cnn}. 
Hence, adding to the translation equivariance \Cref{eq:trans-equiv}, we can construct CNNs equivariant to the chosen discrete rotation group $C_N$. 
Channel-wise activations commute with the orientation relabeling, so equivariance to rotations (and translations) is preserved layer by layer.

% --- Multi-block and pooling ---
A practical architecture uses multiple lifted blocks $v_1,\dots,v_b$; the overall action is the block-diagonal (direct-sum) representation $\pi=\bigoplus_{i=1}^b \pi_i$. In \Model, because we require a vector invariant to rotations and translations, we apply a group pooling and a spacial pooling per block at last; i.e.,
\begin{equation}
    \mathrm{Pool}(v_i)(x) := \max_{h\in C_N} v_i(x)[h], \quad
    \mathrm{Pool}(v_i) := \text{Mean}_{x \in \R^2} v_i(x),
\end{equation}
which yields features invariant to translations and rotations.

\subsection{Navigating \AFiii with inter-domain distance restraints}
\label{subsec:guided-diffusion}

Next, we describe a method for imposing a restraint on the generative process of \AFiii so that the CVs approaches a user-specified target values. 
Specifically, while \AFiii employs a score-based diffusion model, we compute the likelihood of the generated structure with respect to the specified CVs and add to the original \AFiii score a driving force that increases this likelihood. With this modified score, the generated structures becomes explicitly conditioned on the CVs.

Let $a$ denote an amino acid sequence and $X=(x_1,\ldots,x_m)$ the 3D coordinates of all atoms, where $x_i\in\mathbb{R}^3$.
We write $X_t$ for the state at diffusion time $t\in[0,1]$.
Sampling the generative process of \AFiii can be written as the reverse-time variance-preserving SDE (VP-SDE):
\begin{equation}
    dX_t = -\Bigl(\tfrac{1}{2}X_t + \nabla_{X_t} \log \pt{X_t}\Bigr) \beta_t dt + \sqrt{\beta_t} d\bar{W}_t,
    \label{eq:af3-sde}
\end{equation}
where $\beta_t$ is the noise schedule, $\bar{W}_t$ is a standard Wiener process, and the score $\nabla_{X_t}\log \pt{X_t}$ is approximated by a denoising network $s_\theta(X_t,t,a)$. We integrate \Cref{eq:af3-sde} backward from $t=1$ to $t=0$ in the generative process.

Let $\phi_{\text{object}} \in \mathbb{R}^D$ denote the differences in the CVs between the generated structure and the target structure, where $D$ is the dimension of the CVs, and let $\phi: X \to \mathbb{R}^D$ be a differentiable mapping from structures to this feature space.
By Bayes’ rule, the conditional score decomposes as
\begin{equation}
    \nabla_{X_t}\log p_t \bigl(X_t \big| \phi_{\text{object}}, a \bigr)
    = \nabla_{X_t} \log \pt{X_t} + \nabla_{X_t} \log \ptc{\phi_{\text{object}}}{X_t}.
    \label{eq:score-decomp}
\end{equation}
Accordingly, we bias the reverse-time dynamics by adding the guidance term:
\begin{equation}
    dX_t = -\Bigl(\tfrac{1}{2}X_t + \nabla_{X_t}\log \pt{X_t} + \eta_t \nabla_{X_t}\log \ptc{\phi_{\text{object}}}{X_t}\Bigr)\,\beta_t\,dt + \sqrt{\beta_t}\,d\bar{W}_t,
    \label{eq:af3-guided}
\end{equation}
so that the process is guided toward $\phi(X_0) \approx \phi_{\text{object}}$.
Here, $\eta_t$ is a time-dependent guidance strength.

In practice, we replace the guidance score by the negative gradient of a squared-error loss:
\begin{equation}
    \mathcal{L}_{\text{MSE}}(X) \;=\; \bigl\|\phi(X)-\phi_{\text{object}}\bigr\|_2^2,
\end{equation}
\begin{equation}
    \nabla_{X_t} \log \ptc{\phi_{\text{object}}}{X_t} \approx - c_t \nabla_{X_t} \mathcal{L}_{\text{MSE}}(X_t),
\end{equation}
which is consistent with an isotropic Gaussian model $p_t \bigl( \phi_{\text{object}} | X_t,a \bigr) \propto \exp \big( - \| \phi(X_t) - \phi_{\text{object}} \|_2^2 / ( 2 \sigma_t^2 ) \big)$, in which case $c_t = 1/(2\sigma_t^2)$. 
The scale $c_t$ can be absorbed into $\eta_t$, which is analyzed in \Cref{app:guidance_sch}.


\subsection{\Model}
\label{subsec:afm-guidance}

\Model operates in three stages: (1) Preparation, where candidate conformations and pseudo-AFM images are generated for training; (2) Training, where a CNN is trained to predict inter-domain distances from pseudo-AFM images; and (3) Inference, where the trained CNN guides the generation process of \AFiii.

\subsubsection{Preparation}
\label{subsubsec:afm-prep}

As training data for a CNN, pseudo-AFM images are generated analytically from 3D conformations. 
Following \cite{matsunaga2023endtoend}, each conformation is placed on the stage at $z=0$ and scanned with a virtual AFM probe; the pseudo-height image corresponds to the probe’s vertical displacement (i.e., a morphological dilation).

The pipeline consists of: (i) assembling a candidate conformation set that covers practically relevant domain-level arrangements so that major conformations are not missed, and (ii) rendering pseudo-AFM images from those conformations.

\paragraph{Constructing candidate conformations.}
We propose an efficient procedure to cover domain-level rearrangements while avoiding unrealistic outliers that would contaminate the CNN's prediction.
First, a reference structure $X_{\text{ref}}$ is obtained with \AFiii (with no restraints).
%Let $\phi(X)\in\R^D$ denote the previously defined projection to the inter-domain distance space $\R^D$.
Target vectors $\{\phi_{\text{perturb}}^{\,i}\}$ are then placed on a grid around $\phi(X_{\text{ref}})$ such that, for each axis $d$,

\begin{equation}
    \phi_{\text{perturb},d}^i \in \bigl[ 0, 4.0 \times \phi(X_{\text{ref}})_d \bigr], \qquad \forall i.
    \label{eq:target_domain_distance}
\end{equation}

The grid uses a step of $0.6~\mathrm{nm}$ for each axis. 
Then, we generate conformations with \AFiii using $\{\phi_{\text{perturb}}^{\,i}\}$ as the restraints.

\paragraph{Validity criterion and stopping rule.}
A generated conformation is marked \emph{invalid} if fewer than $92\%$ of its C$\alpha$–C$\alpha$ bond lengths fall within $[0.37,0.39]~\mathrm{nm}$.
To avoid unproductive sampling, targets on the $\phi$-grid are processed in order of increasing distance from $\phi(X_{\text{ref}})$, starting at $\phi_{\text{perturb}}=\phi(X_{\text{ref}})$.
For any frontier target, if all candidates within a three-step neighborhood on the grid (i.e., nodes reachable in at most three edges) yield invalid realizations, exploration beyond that target is stopped.

\paragraph{Geometric sanitization.}
Generated conformations are filtered with MolProbity using permissive thresholds (see \cref{tab:score_thresholds}) to remove geometric violations. 
Together, broad coverage in $\phi$-space and MolProbity-based filtering yield a conformation set that is both sufficiently wide and geometrically clean for training the CNN in \Model.

\begin{table}[H]
\centering
\caption{MolProbity thresholds for candidate conformations.}
\label{tab:score_thresholds}
\begin{tabular}{|l|c|}
\hline
Criterion & Threshold \\
\hline
MolProbity Score & $\leq 10.0$ \\
Clash Score & $\leq 13.0$ \\
Ramachandran favored (\%) & $\geq 90.0$ \\
Rotamer Outlier (\%) & $\leq 50.0$ \\
\hline
\end{tabular}
\end{table}

\paragraph{Pseudo-AFM image rendering.}
As preprocessing, each selected conformation is randomly rotated and translated so that its minimum $z$-coordinate equals $0$. 
The $xy$ position is uniformly randomized within the image boundaries while keeping the molecule inside the frame.
Pseudo-AFM images are then generated using the settings in \cref{tab:afm_settings}.
Because the tip geometry is unknown in experiments, the tip radius $r$ and taper angle $a$ are sampled uniformly within specified ranges to mimic realistic variability.
For \flhac, \verb|skimage.exposure.match_histograms| is applied to match each image histogram to that of all experimental frames.

\begin{table}[htbp]
\centering
\caption{Settings for training AFM images.}
\label{tab:afm_settings}
\begin{tabular}{|l|c|c|}
\hline
 & \AK & \flhac \\
\hline
Resolution [nm/pixel] & 0.3 & 0.98 \\
Image Size [pixel $\times$ pixel] & $35 \times 35$ & $35 \times 35$ \\
Tip Radius [nm] & $r \sim \mathrm{Uniform}(1,2)$ & $r \sim \mathrm{Uniform}(2,6)$ \\
Tip Angle [degree] & $a \sim \mathrm{Uniform}(10,30)$ & $a \sim \mathrm{Uniform}(10,30)$ \\
Noise Std. Dev. [nm] & 0.0 & 0.5 \\
Histogram Matching & --- & \checkmark \\
Dataset Size [frames] & 5M & 5M \\
\hline
\end{tabular}
\end{table}

\subsubsection{Training}
\label{subsubsec:afm-training}

We adopted a CNN that is invariant under the discrete version of the $\SE$ group, namely $(\R^2, +) \rtimes C_8$. The specific implementation is described in \cref{alg:cnscnn}. 

All models in this paper were trained on a single node equipped with a NVIDIA RTX A6000 GPU (48~GB memory). 
The training took approximately 5 hours for the \AK dataset, whereas it required about 64 hours for \flhac. 
The loss convergence for \flhac took significantly longer, but our empirical observation suggests that the training time tends to depend on the range of the tip radius and the noise level.

\subsubsection{Inference}
\label{subsubsec:afm-inference}

In inference, the pre-trained CNN predicts $\phi_{\text{predict}}$ from an AFM image, and \AFiii generate a 3D conformation $X_{\text{gen}}$ using $\phi_{\text{predict}}$ as a restraint. 
For both \AK and \flhac, the inference took less than 1 minute.

%In this study, because the guidance scheduling $\eta_t$ was adjusted, the inter-domain distances of generated 
%conformations often deviated from the restraints. To address this issue, instead of directly adopting the CNN's output as $\phi_{\text{target}}$, we set restrictions based on past predictions, typically the conformation pool created during training data generation. 
We considered inference to be successful when the root squared error in the inter-domain distance space between $\phi_{\text{predict}}$ and $\phi(X_{\text{gen}})$ was less than 0.1~nm.

\subsubsection{Evaluation}
\label{subsubsec:afm-evaluation}

To evaluate how well the predicted conformation reproduces the reference image, we performed a rigid-body fitting procedure comprising 60{,}000 random rotations uniformly sampled from $\mathrm{SO}(3)$.
For each rotation $R \in \mathrm{SO}(3)$, we (i) rotated the structure by $R$, (ii) aligned its $xy$ coordinates to the image centroid, and (iii) translated it in both $x$ and $y$ over $[-5,5]$ nm with 0.5 nm increments.

At each translated pose, we generated a pseudo-AFM image and computed its the correlation coefficient (c.c.) with the reference image, using \cref{eq:cc}.
For each rotation, we recorded the pose with the maximum c.c. over translations.
Finally, the pose achieving the overall maximum c.c. across all rotations and translations was taken as the optimal pose.

\begin{equation}
    \mathrm{c.c.}(R)=
    \frac{\sum_{p \in \text{pixels}} H^{\text{(exp)}}_p H^{\text{(sim)}}_p(R)}
         {\sqrt{\sum_{p \in \text{pixels}} \left(H^{\text{(exp)}}_p\right)^2}
          \sqrt{\sum_{p \in \text{pixels}} \left(H^{\text{(sim)}}_p(R)\right)^2}}
    \label{eq:cc}
\end{equation}

