\begin{abstract}
%Direct observation of biomolecules in action is fundamental to elucidating the mechanisms of their activity. High-speed atomic force microscopy (HS-AFM) is a powerful technique that enables direct visualization of protein morphology and dynamics in solution. However, AFM is intrinsically limited to measuring the surface geometry of molecules, which precludes atomic-level structural characterization. Here, we propose \Model, a generative AI-driven approach to reconstruct atomic-level protein structures from AFM images. \Model navigates the sampling trajectory of \AFiii \cite{Abramson2024, bytedance2025protenix} with AFM-derived structural restraints, progressively guiding it towards conformations consistent with experimental AFM data. We evaluated \Model through a twin experiment: using pseudo-AFM images of Adenylate kinase, we first validated our framework and demonstrated that it can reproduce conformations closely resembling the ground truth. Furthermore, we applied \Model to real experimental data of the flagellar protein \flhac and showed that it outperforms rigid-body fitting for reproducing AFM images. Our method enables rapid structure estimation from AFM images, making it possible to process all frames in AFM movies and opening new avenues for the study of protein conformational dynamics.
High-speed atomic force microscopy (HS-AFM) enables direct visualization of protein dynamics under near-physiological conditions, yet its intrinsic limitation to surface topography prevents atomic-level structural characterization. We present AFM-Fold, a generative AI-based framework that reconstructs three-dimensional protein conformations directly from AFM images. AFM-Fold combines a rotation-equivariant convolutional neural network, which extracts low-dimensional collective variables (CVs) from AFM images, with a guided diffusion process that generates conformations consistent with the inferred CVs.
Using pseudo-AFM images of adenylate kinase, AFM-Fold accurately reproduced not only the open and closed conformations, but also intermediate states. Application to 159 experimental HS-AFM frames of the flagellar protein \flhac further demonstrated that AFM-Fold outperforms rigid-body fitting and captures time-correlated domain motions that reflect underlying conformational dynamics.
Importantly, AFM-Fold reconstructs physically plausible structures within approximately one minute per frame without requiring molecular dynamics simulations, making it suitable for high-throughput analysis of HS-AFM movies. By enabling rapid structural interpretation of AFM data, AFM-Fold provides a means to link real-time molecular imaging to underlying conformational landscapes, thereby expanding the biological insight that can be extracted from HS-AFM experiments.
\end{abstract}
