\section{Discussion and Conclusions}

In this work, we have proposed a computational protocol, \Model, to predict the 3D conformation of biomolecules from AFM images. 
\Model enables rapid and accurate conformation estimation compared to previous approaches.
\Model extracts features from AFM images with a group-invariant (symmetry-aware) CNN, reducing computational cost in two ways.
First, it reduces the amount of required training data, because symmetry-aware representations learn efficiently from limited images \cite{e2cnn}.
Second, it shortens inference time: guiding \AFiii requires computing $\ptc{\phi_{\text{object}}}{X_t}$; replacing this with $\mathcal{L}_{\text{MSE}}(X_t)$ keeps the computational cost per structure low.
In the estimations of this study, per-frame estimation finished within one minute on a single GPU.
It has the potential to serve as a powerful tool for interpreting HS-AFM movies comprised of hundreds of frames in both structural and dynamic aspects.

%\paragraph{Merits}
% Candidates: Advantages, Merits, Benefits

%Across our evaluations, \Model delivered competitive prediction accuracy at low computational cost.
%Therefore, \Model can help deepen understanding of biological processes---for example, by quantifying large-scale domain rearrangements along putative pathways and by prioritizing follow-up measurements in single-molecule studies.

As a limitation of \Model, it requires the CVs to be estimated from the AFM image, which is a pre-processing step.
The \Model framework can be readily extended by altering the physical observables it incorporates.
While we used inter-domain distances here, it can incorporate residue-level distance restraints, enabling AFM-guided reconstructions that capture finer-grained motions without a disproportionate increase in runtime.

%\paragraph{Limitations \& Future Work}
% Candidates: Weaknesses, Constraints, Challenges
\Model infers conformations from a single AFM image and currently does not quantify predictive uncertainty due to image noise or molecular orientation.
Extending the framework to estimate uncertainty would make the predictions easier to interpret \cite{Kendall2017Uncertainty}.
Moreover, AFM image sequences constitute time-series data in which poses change only slightly between adjacent frames; incorporating temporal dependencies---for example, with TimeSformer \cite{Bertasius2021TimeSformer} or non-local networks \cite{Wang2018NonLocalNetworks}---could further improve fidelity.
Finally, in this work we steered \AFiii's sampling using AFM images; consequently, neither thermodynamic stability nor explicit physical priors is incorporated.
In the future, conditioning the prior of an ensemble generative model should enable integrated modeling that unifies experimental measurements, physical priors, and computational inference.

\section{Code Availability Statement}
text here

\section{Data Availability Statement}
text here

\section{Acknowledgements}
\label{sec:acknowledgements}
We appreciate Professor Tohru Minamino (The University of Osaka) and Professor Noriyuki Kodera (Kanazawa University) for kindly providing the experimental AFM image data used in this work. This work was supported by MEXT as "Program for Promoting Researches on the Supercomputer Fugaku" (Development and application of large-scale simulation-based inferences for biomolecules JPMXP1020230119) and used computational resources of supercomputer Fugaku provided by the RIKEN Center for Computational Science (Project IDs: hp230209 hp240215, hp250233). 


