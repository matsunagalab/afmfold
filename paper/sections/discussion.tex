\section{Discussion and Conclusions}

In this work, we have proposed a computational protocol, \Model, to predict the 3D conformation of biomolecules from AFM images. 
\Model enables rapid and accurate conformation estimation compared to previous approaches.
\Model extracts features from AFM images with a group-invariant (symmetry-aware) CNN, reducing computational cost in two ways.
First, it reduces the amount of required training data, because symmetry-aware representations learn efficiently from limited images \cite{e2cnn}.
Second, it shortens inference time: guiding \AFiii requires computing $\ptc{\phi_{\text{object}}}{X_t}$; replacing this with $\mathcal{L}_{\text{MSE}}(X_t)$ keeps the computational cost per structure low.
In the estimations of this study, per-frame estimation finished within one minute on a single GPU.
It has the potential to serve as a powerful tool for interpreting HS-AFM movies comprised of hundreds of frames in both structural and dynamic aspects.

%\paragraph{Merits}
% Candidates: Advantages, Merits, Benefits

%Across our evaluations, \Model delivered competitive prediction accuracy at low computational cost.
%Therefore, \Model can help deepen understanding of biological processes---for example, by quantifying large-scale domain rearrangements along putative pathways and by prioritizing follow-up measurements in single-molecule studies.

A limitation of \Model is that it requires a priori definition of CVs as features to be estimated from the AFM image. If it is challenging to define appropriate CVs a priori, one could first perform structural sampling using methods such as AlphaFlow \cite{jing2024alphaflow}, BioEmu \cite{bioemu2025}, or \AFii with MSA subsampling \cite{delAlamo2022sampling}, and then identify meaningful CVs by applying dimensionality reduction techniques like PCA to extract important features. 
The \Model framework is general, and as long as the chosen CVs are differentiable with respect to atomic Cartesian coordinates and can be used for guiding \AFiii, a wide variety of CV types can be incorporated. For example, we focused on inter-domain distances as CVs in this work, ohter CVs, such as residue-level distance restraints can be used, to enable AFM-guided reconstruction of finer-grained motions without a significant increase in computational cost. 

%\paragraph{Limitations \& Future Work}
% Candidates: Weaknesses, Constraints, Challenges
Another limitation of \Model is that it infers conformations from a single AFM image and currently does not quantify predictive uncertainty due to image noise or molecular orientation. 
Extending the framework to estimate uncertainty would make the predictions easier to interpret \cite{Kendall2017Uncertainty}.
Moreover, AFM image sequences constitute time-series data in which poses change only slightly between adjacent frames; incorporating temporal dependencies---for example, with TimeSformer \cite{Bertasius2021TimeSformer} or non-local networks \cite{Wang2018NonLocalNetworks}---could further improve fidelity.

Finally, in this work we steered \AFiii's sampling only using AFM images; consequently, neither structural stability nor explicit physical priors is incorporated. 
In the future, conditioning the prior (such as Boltz-2's custom potentials \cite{passaro2025boltz2}) of an ensemble generative model should enable integrated modeling that unifies experimental measurements, physical priors, and computational inference. 

\section{Code Availability Statement}
The code of \Model is publicly available at http://github.com/matsunagalab/afmfold.

\section{Data Availability Statement}
Notebooks, scripts, and datasets to reproduce the results of this paper are publicly available at Zenodo \cite{kawai_dataset_nodate}. HS-AFM data of \flhac are available upon reasonable request to corresponding author and with the permission of the original authors \cite{inoue_structural_2019}. 

\section{Acknowledgements}
\label{sec:acknowledgements}
We appreciate Tohru Minamino (The University of Osaka) and Noriyuki Kodera (Kanazawa University) for kindly providing the experimental AFM image data used in this work. 
This work was by JSPS KAKENHI (Grant number: 23H03412), and partly supported by MEXT as "Program for Promoting Researches on the Supercomputer Fugaku" (Development and application of large-scale simulation-based inferences for biomolecules JPMXP1020230119) and used computational resources of supercomputer Fugaku provided by the RIKEN Center for Computational Science (Project IDs: hp230209 hp240215, hp250233). 
