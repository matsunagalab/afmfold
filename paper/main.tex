\documentclass{article}

% Language setting
\usepackage[english]{babel}

% Set page size and margins
\usepackage[letterpaper,top=2cm,bottom=2cm,left=3cm,right=3cm,marginparwidth=1.75cm]{geometry}

% Packages
\usepackage{amsmath, amsfonts, amssymb}
\usepackage{amsthm}
\usepackage{graphicx}
\usepackage{algorithm}
\usepackage[noend]{algpseudocode}
\usepackage{xspace}
\usepackage{gensymb}
\usepackage{float}
\usepackage{appendix}
\usepackage{caption}
\captionsetup[table]{skip=3pt}

% リンク付き参照
\usepackage[colorlinks=true, allcolors=blue]{hyperref}
\usepackage[capitalize]{cleveref}
\crefname{figure}{fig.}{figs.}
\Crefname{figure}{Fig.}{Figs.}
\crefname{equation}{eq.}{eqs.}
\Crefname{equation}{Eq.}{Eqs.}
\crefname{table}{table}{tables}
\Crefname{table}{Table}{Tables}
\crefname{algorithm}{alg.}{algs.}
\Crefname{algorithm}{Alg.}{Algs.}
\usepackage{authblk}

% biblatex と相性の良い引用符パッケージ
\usepackage{csquotes}

% 文献:biblatex+biber(数字スタイル)
\usepackage[backend=biber,style=chem-acs,sorting=none,giveninits=true]{biblatex}
\addbibresource{citation.bib}

% 定理環境
\theoremstyle{plain}
\newtheorem{theorem}{Theorem}[section]
\newtheorem{lemma}[theorem]{Lemma}
\theoremstyle{definition}
\newtheorem{definition}[theorem]{Definition}
\theoremstyle{remark}
\newtheorem{remark}[theorem]{Remark}

% カスタム定義のコマンド


% タンパク質
\newcommand{\flhac}{FlhA\textsubscript{C}\xspace}
\newcommand{\AK}{AK\xspace}
\newcommand{\acdi}{A\textsubscript{C}D\textsubscript{1}\xspace}
\newcommand{\acdii}{A\textsubscript{C}D\textsubscript{2}\xspace}
\newcommand{\acdiii}{A\textsubscript{C}D\textsubscript{3}\xspace}
\newcommand{\acdiv}{A\textsubscript{C}D\textsubscript{4}\xspace}

% モデル
\newcommand{\ModelName}[1]{\textrm{#1}\xspace}
\newcommand{\AFii}{\ModelName{AlphaFold~2}}
\newcommand{\AFiii}{\ModelName{AlphaFold~3}}
\newcommand{\Boltzi}{\ModelName{Boltz-1}}
\newcommand{\Boltzii}{\ModelName{Boltz-2}}
\newcommand{\Model}{\ModelName{AFM-Fold}}

% 数式
\newcommand{\R}{\mathbb{R}}
\newcommand{\gcnn}{g\text{-}CNN\xspace}
\newcommand{\gcnns}{g\text{-}CNNs\xspace}
\newcommand{\pt}[1]{p_t(#1 | a)}
\newcommand{\ptc}[2]{p_t(#1 | #2, a)}


\title{Rapid Reconstruction of Atomic 3D Configurations from an AFM image by AlphaFold~3}

\author[1]{Tsuyoshi Kawai}
\author[1,2]{Yasuhiro Matsunaga\footnote{ymatsunaga@riken.jp}}

\affil[1]{Graduate School of Science and Engineering, Saitama University}
\affil[2]{RIKEN Center for Computational Science}

\begin{document}
\maketitle

\begin{abstract}
Direct observation of biomolecules in action is fundamental to elucidating the mechanisms of their activity. High-speed atomic force microscopy (HS-AFM) is a powerful technique that enables direct visualization of protein morphology and dynamics in solution. However, AFM is intrinsically limited to measuring the surface geometry of molecules, which precludes atomic-level structural characterization.
Here, we propose \Model, a generative AI-driven approach to reconstruct atomic-level protein structures from AFM images. \Model navigates the sampling trajectory of \AFiii \cite{Abramson2024, bytedance2025protenix} with AFM-derived structural restraints, progressively guiding it towards conformations consistent with experimental AFM data. 
We evaluated \Model through a twin experiment: using pseudo-AFM images of Adenylate kinase, we first validated our framework and demonstrated that it can reproduce conformations closely resembling the ground truth. Furthermore, we applied \Model to real experimental data of the flagellar protein \flhac and showed that it outperforms rigid-body fitting for reproducing AFM images.
Our method enables rapid structure estimation from AFM images, making it possible to process all frames in AFM movies and opening new avenues for the study of protein conformational dynamics.
\end{abstract}

\section{Introduction}
\label{sec:intro}

High-speed atomic force microscopy (HS-AFM) enables direct visualization of biomolecular dynamics in solution, offering mechanistic insight into how biomolecules change their conformations and its function. However, AFM images report only the surface topography of the specimen, and their spatial resolution is fundamentally constrained by the effective radius of the cantilever tip. Overcoming these limitations requires computational methodologies that can bridge AFM surface imaging with atomistic three-dimensional (3D) structural models.

An early strategy to address this challenge has been rigid-body fitting, in which the rotation and translation of a molecular complex are optimized to best match the AFM image. Although powerful, this approach cannot account for conformational changes. In real biological systems, the function of biomolecules often depends not only on a single static conformation but also on the conformational ensemble. Therefore, characterizing these conformational transitions from AFM images is critical for analyzing and interpreting the experimental AFM data. 

In recent years, methods have been proposed that utilize molecular dynamics (MD) or Monte Carlo (MC) simulations to generate conformations consistent with AFM images. For example, in flexible fitting \cite{Niina2020}, defining the agreement with an AFM image as a potential function $V_{\mathrm{AFM}}$, MD simulations are driven to minimize it. In NMFF-AFM \cite{Wu2024}, optimizing a subset of normal modes effectively prevents overfitting. Although such approaches can be effective, they are constrained by the general limitations inherent in molecular simulations. In the former case, sampling is often trapped in local minima, making optimization slow when separated by high barriers. In the latter case, the identification of normal modes is computationally demanding, which further restricts the practical applicability. As a result, achieving both general applicability and reproducibility remains a significant challenge.

From the perspective of machine learning, the development of sequence-to-structure models such as \AFii \cite{jumper2021highly}, \AFiii, and \Boltzi \cite{wohlwend2024boltz1} have demonstrated unprecedented precision in predicting the 3D fold of a protein. These models can be regarded as generating atomistic conformational ensembles from a given sequence. However, in practice, the resulting conformational distributions typically exhibit sharp peaks around specific stable states and thus fail to capture a diverse range of conformations.

Recently, guiding the generative process of diffusion-based structure prediction models, including \AFiii and \Boltzi, has emerged as an approach aimed at generating targeted conformational ensembles. For example, in the context of NMR \cite{maddipatla2025inverse}, it has been demonstrated that imposing electron density–based restraints during the generative process enables the generation of ensembles consistent with experimental observations. Similarly, for Cryo-EM \cite{raghu2025multiscale}, the incorporation of global or local density-map fitting as restraints has been proposed as a means of steering the predicted conformations toward agreement with the experimental data.

In this study, we introduce \Model, which extends these ideas to AFM by employing structural restraints extracted from AFM images to guide the sampling trajectories of pre-trained protein structure prediction models. Specifically, a group-invariant CNN \cite{e2cnn} estimates low-dimensional structural variables---namely, inter-domain distances---from AFM images, and these estimated values are then used as restraints to guide the generative process, enabling the sampling of 3D conformations consistent with AFM images.

We validated \Model in twin experiments. First, using Adenylate Kinase from Escherichia coli (\AK, PDBID: 1AKE \cite{muller1992adenylate}, 4AKE \cite{muller1996adenylate}), we reconstructed the underlying conformations and accurately reproduced the ground-truth conformations. Second, we applied \Model to experimental data from the flagellar protein \flhac (PDBID: 3A5I \cite{saijo-hamano2010flha}) and confirmed that it predicts conformations more consistent with AFM images than the corresponding PDB database structure.

The main contributions of this study are as follows.
\begin{itemize}
    \item We propose \Model, a novel approach for reconstructing atomic-level protein structures from AFM images.
    \item Using pseudo-AFM images of a model protein, we demonstrated that \Model can accurately reconstruct the ground-truth conformations.
    \item With experimental AFM data, we confirmed that \Model outperforms rigid-body fitting and provided concrete examples of its application.
    \item In contrast to existing structure reconstruction methods from AFM images, which typically require hours to days, our approach enables structure prediction in under a minute while avoiding overfitting.
    \item In the Appendix, we systematically evaluated guidance strength for the generative module and summarized effective strategies. We further proposed a method to sample structures from a broad conformational space at low computational cost by leveraging these strategies.
\end{itemize}

A key aspect of \Model is that learning, generation, and inference are carried out without any prior assumptions about long-timescale MD simulations or slow dynamical modes. This enables accurate prediction while reducing computational cost, including the preparation of training data, and allows the observation of dynamics without force-field biases. \Model thus provides a powerful framework for observing conformational transitions and functional mechanisms underlying AFM images, and the resulting insights will advance the understanding of biomolecular mechanisms, and also improve MD simulations themselves.

\begin{figure}[htbp]
    \centering
    \includegraphics[width=\linewidth]{images/figures-01.png}
    \caption{
        \textbf{Schematic overview of \Model and outcome}. AFM-based guidance steers \AFiii and reproduces the closed and open states of \AK.
        \textbf{(a)} Schematic diagram of \Model. 
        The AFM image, shown at the lower left, serves as the input and is processed as follows: 
        (1) coordinates in the inter-domain distance space are predicted from the AFM image by a group-invariant CNN; 
        (2) at diffusion time $t > 0.54$, the mean squared error (MSE) is computed between the predicted coordinates and the generated conformations; 
        (3) the gradient of the MSE is used as a correction score to guide the generative process of \AFiii. 
        \textbf{(b)(c)} Results of structure prediction from pseudo-AFM images of \AK. 
        In (c), the blue structures (left: close; right: open) represent the reference ground-truth conformations, which were used to generate the two reference images shown at the bottom of (b). 
        Using these reference images, our method successfully predicted the red structures in (c). 
        After aligning the predicted conformations to the reference conformations by rigid-body fitting, the resulting pseudo-AFM images are shown at the top of (b), demonstrating good agreement with the reference images.
        }
    \label{fig:overview}
\end{figure}

\section{Related Work}
\label{sec:related}

To estimate 3D structures from AFM images, approaches such as flexible fitting \cite{Niina2020} and NMFF-AFM \cite{Wu2024} find conformations that reproduce images with a high correlation coefficient (c.c.) with the reference AFM image. While effective, these methods depend on long simulation trajectories and thus suffer from high computational cost. 

On the other hand, generative AI-based approaches employing diffusion models or flow models enable us to efficiently  sample molecular conformations in multiscale manners. For example, recent structure-generation foundation models, including \AFiii and \Boltzi, employ diffusion-based formulations. In particular, \Boltzii \cite{passaro2025boltz2} introduced \emph{boltz-steering}, a framework that imposes a customized potential to steer the sampling towards physically plausible conformations. Nevertheless, these models largely follow a ``one sequence--one structure'' paradigm, which makes it difficult to produce diverse conformational ensembles for a single amino-acid sequence. 

Several approaches have been proposed to address this problem. For example, the MSA subsampling technique \cite{delAlamo2022sampling} and its extensions \cite{Wayment-Steele2023MSAClustering, daSilva2024_subsampledAF2, KalakotiWallner2025_AFsample2} increase the diversity of multiple sequence alignments (MSAs) by masking parts of the MSA. While these approaches can improve diversity in certain systems, the relationship between perturbations in the MSA and the resulting structural diversity would be unclear. The observed structural diversity is not guaranteed to follow the thermodynamic stability dictated by the real Boltzmann distribution.

Another powerful direction is the development of ensemble generative or MD surrogate models. An early contribution in this area is the Boltzmann generator \cite{NoeOlssonKohlerWu2019_BoltzmannGenerators}, which was designed to learn and reproduce the Boltzmann distribution of small, specific peptides. More recently, models have been developed to generate diverse structural ensembles from sequences by learning from extensive MD simulation data across a wide range of proteins \cite{jing2023eigenfold, jing2024alphaflow, Zheng2024_DiG_NatMachIntell, Cheng2024_AlphaFolding, Jing2024mdgen, jin2025p2dflow, bioemu2025}. However, obtaining sufficiently long MD trajectories for training requires substantial computational resources, and such models inevitably inherit the force-field biases present in the training data. One promising alternative is to leverage experimental data as training data sets, which offers a pathway to develop ensemble generators that are grounded in experimental observations rather than computational approximations.

To mitigate these issues, recent studies have explored conditioning ensemble generative models on experimental measurements or physical prior knowledge. Examples include methods that generate NMR-consistent structural ensembles \cite{maddipatla2025inverse, Liu2024_EGDiff, Liu2025_ExEnDiff}, and cryoEM-consistent modelig using \AFiii, \Boltzi, and Chroma \cite{raghu2025multiscale, levy2025solving}. Following this direction, \Model enables AFM-conditioned conformation generation, steering \AFiii's prior distribution mode towards modeling more dynamic conformatinoal events captures by the high-speed AMF images.

\section{Methods}
\label{sec:methods}

Our \Model framework leverages \AFiii (here, we used the open-sourced Protenix \cite{bytedance2025protenix} model, a PyTorch reimplementation of \AFiii) to estimate the underlying 3D conformation from a given AFM image. To generate 3D structures consistent with AFM images using \AFiii, we follow two main steps: (i) First, we estimate low-dimensional collective variables (CVs) for the target molecule from the given AFM image using a pre-trained convolutional neural network (CNN). For CVs, we consider the use of inter-domain distances, assuming multi-domain proteins as a typical case. (ii) Next, we use the estimated CVs to add restraints to \AFiii during structure generation, producing 3D structures that are consistent with the AFM image (see \Cref{fig:overview}a).

%First, the reference AFM image is fed into a convolutional neural network (CNN), which predicts the inter-domain distances of the molecule captured in the image. 
%Then, by incorporating these predicted distances as restraints into the generation process of \AFiii, a structure consistent with the reference AFM image is obtained (see \cref{fig:overview} (a)).

In this section, we present our methodology in three parts. 
In \cref{subsec:se2cnn}, we review the group-invariant CNN used to estimate the CVs from AFM images. 
In \cref{subsec:guided-diffusion}, we explain how we add restraints in the diffusion process of \AFiii using the estimated CVs. 
Finally in \cref{subsec:afm-guidance}, the overall computational protocol of our \Model framework is given.

\subsection{Group-equivariant CNNs}
\label{subsec:se2cnn}

% --- Setup/context ---
%Unlike many other problems, reconstructing 3D conformations from AFM images requires verifying whether the 3D structures generated by \AFiii are consistent with a given 2D reference image. The generation process of \AFiii is guided to maximize the resulting likelihood, thereby producing conformations that agree closely with AFM images (as explained in \cref{subsec:guided-diffusion}). However, because the orientation of the molecule in the reference image is unknown and may correspond to any element of the full 3D rotation group, a naive search for the ground-truth pose is computationally prohibitive. Thus, an effective strategy is needed to assess the agreement between 2D AFM images and 3D molecular conformations. In particular, rather than transforming conformations into the AFM image format—--which is computationally demanding due to the rotational degrees of freedom—--it is more practical to derive features from AFM images that are more readily accessible from conformational side. 

In AFM 2D image analysis, similar to single particle analysis of cryoEM micrographs, the pose of the 3D structure is unknown. The alignment calculation between a 2D  image and the 3D structural pose requires exhaustive search and is computationally expensive. Moreover, if misalignment occurs, it leads to extreme deterioration in estimation accuracy. In the case of cryoEM, the influence of misalignment can be mitigated on average by a vast number of images, but in the case of AFM, since we want to perform structural estimation for each single image, misalignment cannot be tolerated. Therefore, we decided to use a rotation-equivariant CNN model (\emph{group-equivariant} CNN; g-CNN) \cite{CohenWelling2016_GCNN, CohenWelling2017_SteerableCNNs, e2cnn, WeilerEtAl2018_3DSteerableCNNs} that directly estimates structure-related CVs from 2D AFM images without performing alignment calculations. 

In steering the structure generation process of \AFiii to generate 3D structures consistent with AFM images, it would be possible to optimize by calculating image similarity each time, as done in rigid-body fitting. In this case, higher image reproduction is expected, but on the other hand, we empirically found that not only does computational cost increase, but overfitting to images causes rotamer outliers and atomic clashes, compromising the physical validity of the structure. Therefore, in this study, we decided that our g-CNN model estimates low-dimensional CVs (such as inter-domain distances in the case of multi-domain proteins) and uses them for structure generation in \AFiii.

With this motivation in mind, we summarize the theory of g-CNN (\gcnn) \cite{CohenWelling2016_GCNN, CohenWelling2017_SteerableCNNs, e2cnn, WeilerEtAl2018_3DSteerableCNNs} based approach that enables efficient comparison between AFM images and conformations. A standard CNN is naturally \emph{translation equivariant} but not rotation equivariant. In contrast, \gcnn modifies feature indexing and kernel weight sharing so that the entire network becomes equivariant with respect to a chosen rotation group. With this architecture, the same physical state yields an identical representation regardless of its placement on the image plane, thereby reducing the need for extensive data augmentation and shortening training time. Consequently, we employ the \gcnn to estimate CVs from a single AFM image.

%\paragraph{Translation equivariance}
We model an AFM image as a single-channel (scalar) field $I:\mathbb{R}^2\to\mathbb{R}$ with pixel coordinate $x=(x_1,x_2)\in\mathbb{R}^2$. 
In a standard CNN, intermediate feature maps are represented as functions $f:\mathbb{R}^2 \to \mathbb{R}^c$.
The input image is first mapped to a $c$-channel field by convolution with channel-specific filters $\{\psi_i\}_{i=1}^c$:
\begin{equation}
    (\psi * I)(x)[i] := \int_{\mathbb{R}^2} \psi_i(y)\, I(x - y)\, dy,
    \label{eq:lifting}
\end{equation}
where $*$ is convolution and $dy$ denotes the area element. 
In practice, the integral is implemented as a finite sum over pixels; we keep the continuous notation for clarity.

% --- Group and action for standard fields ---
Here, we consider the planar motion group $\R^2 \rtimes C_N$, where $C_N=\{e,r,\dots,r^{N-1}\}\subset \mathrm{SO}(2)$ is a finite rotation subgroup.
For scalar or standard $c$-channel fields (without group indexing that we later explain), 
the action $\pi(t,g)$ of translation $t\in\R^2$ and rotation $g\in C_N$ is $(\pi(t,g)u)(x)\;:=\;u\big(g^{-1}(x-t)\big)$.

Because convolution commutes with translations, a standard CNN is translation-equivariant. 
For a single convolution with a (matrix-valued) kernel $k(y)\in\R^{c_{\mathrm{out}}\times c_{\mathrm{in}}}$,
\begin{equation}
    \begin{aligned}
        \big(k * (\pi(t,e) u)\big)(x) 
        &= \int_{\mathbb{R}^2} k(y)\, u(x-y-t)\, dy \\
        &= (k * u)(x - t) = \big(\pi(t,e) (k * u)\big)(x).
    \end{aligned}
    \label{eq:trans-equiv}
\end{equation}
Since a typical activation $\sigma$ is pointwise, $\sigma$ is also translation-equivariant, $\sigma (\pi(t, e) u) = \pi(t, e) \sigma (u)$ and the property propagates through the network. By contrast, the standard convolution is generally \emph{not} rotation-equivariant.

%\paragraph{Rotation equivariance}
The key idea of \gcnns is to \emph{lift} an image to a feature field indexed by group elements and to impose weight sharing consistent with the group action.
Given a base filter $\psi:\mathbb{R}^2\to\mathbb{R}$, define its rotated copies $\psi_h(y):=\psi(h^{-1}y)$ for $h\in C_N$, and set
\begin{equation}
    v(x)[h] \;:=\; \int_{\mathbb{R}^2} \psi(h^{-1}y)\, I(x-y)\, dy,
    \qquad h\in C_N.
    \label{eq:gcnn-lifting}
\end{equation}
Thus $v:\mathbb{R}^2\times C_N\to\mathbb{R}$ is a group-indexed (orientation-channel) field.

% --- Equivariance statement (concise) ---
For such lifted fields, if the kernel satisties the following constraint about sharing weights, $k(gy)[h,h'] = k(y) \big[g^{-1}h, g^{-1}h'\big]$ for all $g, h, h' \in C_N$, then the group convolution commutes with $\pi(0,g)$:
\begin{equation}
    \big(k * (\pi(0,g)v)\big)(x)[h] \;=\; \big(\pi(0,g)(k * v)\big)(x)[h].
    \label{eq:gcnn-rot-equiv}
\end{equation}
For details, see \cite{e2cnn}. 
Hence, adding to the translation equivariance \Cref{eq:trans-equiv}, we can construct CNNs equivariant to the chosen discrete rotation group $C_N$. 
Channel-wise activations commute with the orientation relabeling, so equivariance to rotations (and translations) is preserved layer by layer.

% --- Multi-block and pooling ---
A practical architecture uses multiple lifted blocks $v_1,\dots,v_b$; the overall action is the block-diagonal (direct-sum) representation $\pi=\bigoplus_{i=1}^b \pi_i$. In \Model, because we require a vector invariant to rotations and translations, we apply a group pooling and a spacial pooling per block at last; i.e.,
\begin{equation}
    \mathrm{Pool}(v_i)(x) := \max_{h\in C_N} v_i(x)[h], \quad
    \mathrm{Pool}(v_i) := \text{Mean}_{x \in \R^2} v_i(x),
\end{equation}
which yields features invariant to translations and rotations.

\subsection{Navigating \AFiii with inter-domain distance restraints}
\label{subsec:guided-diffusion}

Next, we describe a method for imposing a restraint on the generative process of \AFiii so that the CVs approaches a user-specified target values. 
Specifically, while \AFiii employs a score-based diffusion model, we compute the likelihood of the generated structure with respect to the specified CVs and add to the original \AFiii score a driving force that increases this likelihood. With this modified score, the generated structures becomes explicitly conditioned on the CVs.

Let $a$ denote an amino acid sequence and $X=(x_1,\ldots,x_m)$ the 3D coordinates of all atoms, where $x_i\in\mathbb{R}^3$.
We write $X_t$ for the state at diffusion time $t\in[0,1]$.
Sampling the generative process of \AFiii can be written as the reverse-time variance-preserving SDE (VP-SDE):
\begin{equation}
    dX_t = -\Bigl(\tfrac{1}{2}X_t + \nabla_{X_t} \log \pt{X_t}\Bigr) \beta_t dt + \sqrt{\beta_t} d\bar{W}_t,
    \label{eq:af3-sde}
\end{equation}
where $\beta_t$ is the noise schedule, $\bar{W}_t$ is a standard Wiener process, and the score $\nabla_{X_t}\log \pt{X_t}$ is approximated by a denoising network $s_\theta(X_t,t,a)$. We integrate \Cref{eq:af3-sde} backward from $t=1$ to $t=0$ in the generative process.

Let $\phi_{\text{object}} \in \mathbb{R}^D$ denote the differences in the CVs between the generated structure and the target structure, where $D$ is the dimension of the CVs, and let $\phi: X \to \mathbb{R}^D$ be a differentiable mapping from structures to this feature space.
By Bayes’ rule, the conditional score decomposes as
\begin{equation}
    \nabla_{X_t}\log p_t \bigl(X_t \big| \phi_{\text{object}}, a \bigr)
    = \nabla_{X_t} \log \pt{X_t} + \nabla_{X_t} \log \ptc{\phi_{\text{object}}}{X_t}.
    \label{eq:score-decomp}
\end{equation}
Accordingly, we bias the reverse-time dynamics by adding the guidance term:
\begin{equation}
    dX_t = -\Bigl(\tfrac{1}{2}X_t + \nabla_{X_t}\log \pt{X_t} + \eta_t \nabla_{X_t}\log \ptc{\phi_{\text{object}}}{X_t}\Bigr)\,\beta_t\,dt + \sqrt{\beta_t}\,d\bar{W}_t,
    \label{eq:af3-guided}
\end{equation}
so that the process is guided toward $\phi(X_0) \approx \phi_{\text{object}}$.
Here, $\eta_t$ is a time-dependent guidance strength.

In practice, we replace the guidance score by the negative gradient of a squared-error loss:
\begin{equation}
    \mathcal{L}_{\text{MSE}}(X) \;=\; \bigl\|\phi(X)-\phi_{\text{object}}\bigr\|_2^2,
\end{equation}
\begin{equation}
    \nabla_{X_t} \log \ptc{\phi_{\text{object}}}{X_t} \approx - c_t \nabla_{X_t} \mathcal{L}_{\text{MSE}}(X_t),
\end{equation}
which is consistent with an isotropic Gaussian model $p_t \bigl( \phi_{\text{object}} | X_t,a \bigr) \propto \exp \big( - \| \phi(X_t) - \phi_{\text{object}} \|_2^2 / ( 2 \sigma_t^2 ) \big)$, in which case $c_t = 1/(2\sigma_t^2)$. 
The scale $c_t$ can be absorbed into $\eta_t$, which is analyzed in \Cref{app:guidance_sch}.


\subsection{\Model}
\label{subsec:afm-guidance}

\Model operates in three stages: (1) Preparation, where candidate conformations and pseudo-AFM images are generated for training; (2) Training, where a CNN is trained to predict inter-domain distances from pseudo-AFM images; and (3) Inference, where the trained CNN guides the generation process of \AFiii.

\subsubsection{Preparation}
\label{subsubsec:afm-prep}

As training data for a CNN, pseudo-AFM images are generated analytically from 3D conformations. 
Following \cite{matsunaga2023endtoend}, each conformation is placed on the stage at $z=0$ and scanned with a virtual AFM probe; the pseudo-height image corresponds to the probe’s vertical displacement (i.e., a morphological dilation).

The pipeline consists of: (i) assembling a candidate conformation set that covers practically relevant domain-level arrangements so that major conformations are not missed, and (ii) rendering pseudo-AFM images from those conformations.

\paragraph{Constructing candidate conformations.}
We propose an efficient procedure to cover domain-level rearrangements while avoiding unrealistic outliers that would contaminate the CNN's prediction.
First, a reference structure $X_{\text{ref}}$ is obtained with \AFiii (with no restraints).
%Let $\phi(X)\in\R^D$ denote the previously defined projection to the inter-domain distance space $\R^D$.
Target vectors $\{\phi_{\text{perturb}}^{\,i}\}$ are then placed on a grid around $\phi(X_{\text{ref}})$ such that, for each axis $d$,

\begin{equation}
    \phi_{\text{perturb},d}^i \in \bigl[ 0, 4.0 \times \phi(X_{\text{ref}})_d \bigr], \qquad \forall i.
    \label{eq:target_domain_distance}
\end{equation}

The grid uses a step of $0.6~\mathrm{nm}$ for each axis. 
Then, we generate conformations with \AFiii using $\{\phi_{\text{perturb}}^{\,i}\}$ as the restraints.

\paragraph{Validity criterion and stopping rule.}
A generated conformation is marked \emph{invalid} if fewer than $92\%$ of its C$\alpha$–C$\alpha$ bond lengths fall within $[0.37,0.39]~\mathrm{nm}$.
To avoid unproductive sampling, targets on the $\phi$-grid are processed in order of increasing distance from $\phi(X_{\text{ref}})$, starting at $\phi_{\text{perturb}}=\phi(X_{\text{ref}})$.
For any frontier target, if all candidates within a three-step neighborhood on the grid (i.e., nodes reachable in at most three edges) yield invalid realizations, exploration beyond that target is stopped.

\paragraph{Geometric sanitization.}
Generated conformations are filtered with MolProbity using permissive thresholds (see \cref{tab:score_thresholds}) to remove geometric violations. 
Together, broad coverage in $\phi$-space and MolProbity-based filtering yield a conformation set that is both sufficiently wide and geometrically clean for training the CNN in \Model.

\begin{table}[H]
\centering
\caption{MolProbity thresholds for candidate conformations.}
\label{tab:score_thresholds}
\begin{tabular}{|l|c|}
\hline
Criterion & Threshold \\
\hline
MolProbity Score & $\leq 10.0$ \\
Clash Score & $\leq 13.0$ \\
Ramachandran favored (\%) & $\geq 90.0$ \\
Rotamer Outlier (\%) & $\leq 50.0$ \\
\hline
\end{tabular}
\end{table}

\paragraph{Pseudo-AFM image rendering.}
As preprocessing, each selected conformation is randomly rotated and translated so that its minimum $z$-coordinate equals $0$. 
The $xy$ position is uniformly randomized within the image boundaries while keeping the molecule inside the frame.
Pseudo-AFM images are then generated using the settings in \cref{tab:afm_settings}.
Because the tip geometry is unknown in experiments, the tip radius $r$ and taper angle $a$ are sampled uniformly within specified ranges to mimic realistic variability.
For \flhac, \verb|skimage.exposure.match_histograms| is applied to match each image histogram to that of all experimental frames.

\begin{table}[htbp]
\centering
\caption{Settings for training AFM images.}
\label{tab:afm_settings}
\begin{tabular}{|l|c|c|}
\hline
 & \AK & \flhac \\
\hline
Resolution [nm/pixel] & 0.3 & 0.98 \\
Image Size [pixel $\times$ pixel] & $35 \times 35$ & $35 \times 35$ \\
Tip Radius [nm] & $r \sim \mathrm{Uniform}(1,2)$ & $r \sim \mathrm{Uniform}(2,6)$ \\
Tip Angle [degree] & $a \sim \mathrm{Uniform}(10,30)$ & $a \sim \mathrm{Uniform}(10,30)$ \\
Noise Std. Dev. [nm] & 0.0 & 0.5 \\
Histogram Matching & --- & \checkmark \\
Dataset Size [frames] & 5M & 5M \\
\hline
\end{tabular}
\end{table}

\subsubsection{Training}
\label{subsubsec:afm-training}

We adopted a CNN that is invariant under the discrete version of the $\SE$ group, namely $(\R^2, +) \rtimes C_8$. The specific implementation is described in \cref{alg:cnscnn}. 

All models in this paper were trained on a single node equipped with a NVIDIA RTX A6000 GPU (48~GB memory). 
The training took approximately 5 hours for the \AK dataset, whereas it required about 64 hours for \flhac. 
The loss convergence for \flhac took significantly longer, but our empirical observation suggests that the training time tends to depend on the range of the tip radius and the noise level.

\subsubsection{Inference}
\label{subsubsec:afm-inference}

In inference, the pre-trained CNN predicts $\phi_{\text{predict}}$ from an AFM image, and \AFiii generate a 3D conformation $X_{\text{gen}}$ using $\phi_{\text{predict}}$ as a restraint. 
For both \AK and \flhac, the inference took less than 1 minute.

%In this study, because the guidance scheduling $\eta_t$ was adjusted, the inter-domain distances of generated 
%conformations often deviated from the restraints. To address this issue, instead of directly adopting the CNN's output as $\phi_{\text{target}}$, we set restrictions based on past predictions, typically the conformation pool created during training data generation. 
We considered inference to be successful when the root squared error in the inter-domain distance space between $\phi_{\text{predict}}$ and $\phi(X_{\text{gen}})$ was less than 0.1~nm.

\subsubsection{Evaluation}
\label{subsubsec:afm-evaluation}

To evaluate how well the predicted conformation reproduces the reference image, we performed a rigid-body fitting procedure comprising 60{,}000 random rotations uniformly sampled from $\mathrm{SO}(3)$.
For each rotation $R \in \mathrm{SO}(3)$, we (i) rotated the structure by $R$, (ii) aligned its $xy$ coordinates to the image centroid, and (iii) translated it in both $x$ and $y$ over $[-5,5]$ nm with 0.5 nm increments.

At each translated pose, we generated a pseudo-AFM image and computed its the correlation coefficient (c.c.) with the reference image, using \cref{eq:cc}.
For each rotation, we recorded the pose with the maximum c.c. over translations.
Finally, the pose achieving the overall maximum c.c. across all rotations and translations was taken as the optimal pose.

\begin{equation}
    \mathrm{c.c.}(R)=
    \frac{\sum_{p \in \text{pixels}} H^{\text{(exp)}}_p H^{\text{(sim)}}_p(R)}
         {\sqrt{\sum_{p \in \text{pixels}} \left(H^{\text{(exp)}}_p\right)^2}
          \sqrt{\sum_{p \in \text{pixels}} \left(H^{\text{(sim)}}_p(R)\right)^2}}
    \label{eq:cc}
\end{equation}

\subsection{Molecular dynamics simulation}
\label{subsec:md}

To verify the \Model framework, we performed MD simulations of \AK and \flhac. 
For \AK, we performed an all-atom MD simulation to compare the distribution of conformational states sampled by MD with those generated by an \AFiii clone (Protenix \cite{bytedance2025protenix}) for training the g-CNN. Simulations were conducted using the Amber ff14SB force field \cite{maier_ff14sb_2015} with the TIP3P-FB water model and ions \cite{wang_building_2014}, starting from the open crystal structure (PDB ID: 4AKE). The simulations were carried out at 300~K and 1~atm using OpenMM \cite{eastman_openmm_2024}, and the trajectory was extended to 450~ns to ensure sufficient sampling of the conformational space.

For \flhac, due to computational time constraints, we did not conduct all-atom MD simulations. Instead, we generated conformational ensembles using a set of MD surrogate models: AlphaFlow \cite{jing2024alphaflow} and BioEmu \cite{bioemu2025}, as well as \AFii with MSA subsampling \cite{delAlamo2022sampling} using ColabFold \cite{mirdita2022colabfold}. These generative approaches enabled us to obtain a diverse ensemble of conformations, facilitating a comprehensive comparison with the distribution generated by our \Model framework.



\section{Results}
\label{sec:results}

\subsection{Verification using pseudo-AFM images: Adenylate Kinase}
\label{subsec:ak_results}

Adenylate Kinase (\AK) is a well-studied monomeric enzyme known for its large-scale conformational transitions. It is composed of three relatively rigid domains: the central CORE domain (residues 1–29, 68–117, and 161–214), the AMP-binding domain (AMPbd; residues 30–67), and the lid-like ATP-binding domain (LID; residues 118–167). Experimental and computatinoal studies have suggested that upon ligand binding, the enzyme undergoes a transition from an inactive open conformation to an active closed conformation (see \Cref{fig:overview}~(c)). Here, we evaluated how accurately \Model reconstructs 3D conformations of \AK from artificially created pseudo ``experimental'' AFM images. 

\paragraph{Quality of the conformational ensemble generated by \AFiii}
First, we checked the quality of the conformational ensemble generated by \AFiii used for the training of the g-CNN. Here, the conformations were generated by steering \AFiii with CV values on a grid as described in \Cref{subsubsec:afm-prep}. As the CVs, we used the inter-domain distances of all possible combinations of domain pairs (LID--CORE, CORE--AMPbd, and AMPbd--LID). To obtain physically plausible structures as accurately as possible via steering to the target CVs, we systematically evaluated generated structures using various steering schedules with different strengths and onset timings, and selected an optimal schedule strategy (see Supporting Information S1 for details). \Cref{fig:ak_cnn_training}~(b) shows the generated conformations projected to the LID--CORE, and CORE--AMPbd distances (colored by the AMPbd--LID distance), in comparison with \Cref{fig:ak_cnn_training}~(a), which shows the projection of the conformations sampled by a 450~ns MD simulation. Compared with the MD simulation, the conformations generated by \AFiii adequately cover a broad conformational space of \AK,  including both closed (PDB ID: 1AKE) and open (PDB ID: 4AKE) crystal structures. This means that the g-CNN trained on this ensemble can learn the broad relationship between AFM images and the inter-domain distances of \AK.

\paragraph{Accuracy of the g-CNN}
Next, we investigated whether the trained g-CNN accurately estimates CV values from given pseudo-AFM images. Here, we estimated CV values from pseudo-AFM images emulated using randomly chosen conformations from the MD simulation trajectory. The pseudo-AFM images were generated with the settings described in \Cref{tab:afm_settings}. \Cref{fig:ak_cnn_training}~(c) shows the distribution of the root squared error in the CV space between the estimated values with the trained g-CNN and the ground-truth values. With the exceptions of some outlier errors greater than 0.500~nm, the trained g-CNN achieves the average squared error of 0.254~nm, which is enough resolution to capture intermediate conformational states of \AK. 

\paragraph{Accuracy of \AFiii steering}
Then, we investigated the accuracy of the estimated conformations of our \Model framework. As a demonstration, we first estimated the closed and open conformations of \AK from the pseudo-AFM images emulated from the closed and open crystal structures (the images are shown in \Cref{fig:overview}~(b)), respectively. \Cref{fig:overview}~(c) show the estimated structures by the protocol of \Model (i.e., g-CNN and \AFiii steering). The RMSD between the ground-truth and reconstructed conformations was 0.216~nm for the closed form and 0.176~nm for the open form, respectively. Considering that the RMSD between the closed and open crystal structures is 0.715~nm, \Model estimation is accurate enough to identify the two states. The c.c. between the AFM images created from the estimated conformations and the given ``experimental'' pseudo-AFM image is 0.997 for the closed form and 0.998 for the open form, respectively (here, \Cref{eq:cc} was used to compute the c.c.). 

We then evaluated the RSMDs of the conformations estimated from the pseudo-AFM emulated from radomly chosen confrmations from the MD simulation trajectory. \Cref{fig:ak_cnn_training}~(e) shows the distribution of the RMSDs between the ground-truth and estimated conformations. Overall, the RMSD values are low and, aside from a small number of outliers, most RMSDs are distributed around the mean value of approximately 0.281~nm. \Cref{fig:ak_cnn_training}~(d) shows the scatter plot of the RMSD and the root squared errors of the predicted CVs between the ground-truth and estimated conformations. Although a reasonable correlation is observed between the two measures, some scatter was observed in RMSD at similar squared error levels; that is, even with comparable CV errors, the RMSDs can vary. Upon close inspection of the generated structures, we found that this arises because a given set of current CVs (inter-domain distances) does not always uniquely determine the domain arrangement. Specifically, either hinge-bending (open--close transition) or shear-like (twisted) domain motions can yield similar inter-domain distances, despite resulting in structurally distinct conformations. This degeneracy in the CV--structure mapping explains the occurrence of RMSD outliers even when the predicted CV values are accurate.

%A second factor that degrades reconstruction accuracy is the quality of the images themselves.
%\Cref{fig:ak_accuracy}~(a) shows the distribution of RMSD between the ground-truth and predicted conformations.
%While most cases exhibit low RMSD, a subset shows high RMSD in \Cref{fig:ak_accuracy}~(a).
%\Cref{fig:ak_accuracy}~(b) shows the distribution of the c.c. between the reference images and the %pseudo-AFM images generated by rigid-body fitting of the predictions to the reference.
%In contrast to RMSD, \Cref{fig:ak_accuracy}~(b) is skewed toward high c.c.
%Indeed, \Cref{fig:ak_accuracy}~(c) indicates that even when RMSD is relatively high, c.c. can remain high (upper-right region), suggesting that some of the reference AFM image does not sufficiently determine inter-domain distances.
%Taken together, despite the presence of systematic errors stemming from methodological limitations of \Model, the framework reliably distinguishes large-scale domain motions with high probability overall.

\begin{figure}[H]
    \centering
    \includegraphics[width=0.7\linewidth]{images/figures-02.png}
    \caption{
        \textbf{Training data and estimation accuracy of \Model}.
        \textbf{(a)} Projection of a 450 ns MD simulation trajectory into the inter-domain distance space. 
        \textbf{(b)} Projection of \AFiii-generated training data into the inter-domain distance space. 
        \textbf{(c)} Density of root squared error in the inter-domain distance space between the trained g-CNN's estimations and the ground-truth. The inter-domain distance was estimated by the trained g-CNNfrom pseudo-AFM images, which were generated from randomly selected MD conformations. The average of the root squared errors is indicated by the broken line (0.254 nm). 
        \textbf{(d)} Scatter plot of the root squared error in the inter-domain distance space and the heavy-atom RMSDs from the ground-truth conformation. 
        The heavy-atom RMSD was computed between the conformation generated with \AFiii steering using the predicted CV values, and the ground-truth conformation.
        \textbf{(e)} Density of the heavy-atom RMSDs. The average of the RMSD is indicated by the broken line (0.281 nm).
        }
    \label{fig:ak_cnn_training}
\end{figure}

\paragraph{Correlation coefficient and robustness against noise}
%So far, we have evaluated the estimation accuracy of \Model using noise-free pseudo-AFM images. To assess whether our approach is applicable to real experimental data, we next investigated the prediction accuracy using pseudo-AFM images with added noise.
%From \cref{fig:noise_robustness}, we observed the following. 
To investigate the applicability of our approach to real experimental data, we next focused on two aspects: (i) the development of evaluation metrics that can be computed without ground-truth structures and (ii) the assessment of robustness in structure estimation accuracy against noise.
As for the first aspect, following the previous study \cite{Niina2020}, we propose to use the c.c. between the given experimental image and a pseudo-AFM image created by rigid-body fitting of the estimated structure to the experimental image maximizing the c.c. values.
\Cref{fig:noise_robustness}~(a) shows the density of the c.c. between the ``experimental'' images (pseudo-AFM images created from randomly selected structures from the MD data) and pseudo-AFM images created by rigid-body fitting of the estimated structures. With the exceptions of some outlier values, the distribution is centered around the average value of 0.998, indicating that the estimated structures are in good agreement with the given images. \Cref{fig:noise_robustness}~(b) shows the scatter plot of the c.c. and the RMSD. 
A clear negative correlation is observed between the c.c. and the RMSD, suggesting that, even when ground-truth structures are unavailable and RMSD cannot be computed, as is often the case with real experimental data, the comparison of c.c.\ values provides a useful indication of the reliability of the estimated structures.

Next, we focused on the second aspect, the assessment of robustness in structure estimation accuracy against noise. So far, we have used noise-free pseudo-AFM images for both the training and inference stages. To assess the robustness of the estimation against noise, we firstly added noise to ``experimental'' pseudo-AFM data in the inference stage using Gaussian noise with a standard deviation of 0.3 nm (which is a typical noise level for HS-AFM images \cite{Niina2020}). 
\Cref{fig:noise_robustness}~(c) shows the scatter plot of the c.c. and the RMSD using the noisy ``experimental'' pseudo-AFM images. Still, a clear negative correlation is observed between the c.c. and the RMSD, suggesting that the c.c. values would work even when noisy real experimental data are used.

Furthermore, for systematic evaluation of the robustness against various noise levels, 
we systematically varied the noise levels added to both the training and the ``experimental'' pseudo-AFM data. \Cref{fig:noise_robustness}~(d) illustrates examples of pseudo-AFM images at different noise levels, and \Cref{fig:noise_robustness}~(e) shows a heat map of the mean RMSD between the estimated and ground-truth structures over the range of noise levels. Notably, we found that adding noise to the ``experimental'' pseudo-AFM images consistently led to substantial reductions in prediction accuracy. In contrast, adding noise to the training data alone had a relatively minor effect on the estimation accuracy; indeed, the best performance was achieved when noise-free training data were used. This indicates that, while robustness to noise in the inputs (``experimental'' images) is limited, the presence or absence of noise in the training data does not substantially affect accuracy, with noise-free training data providing the most reliable results. \Cref{fig:noise_robustness}~(f) shows a heat map of the mean c.c. between the rigid-body fitted pseudo-AFM images and the ``experimental'' pseudo-AFM images under the same noise levels with the \Cref{fig:noise_robustness}~(e).

%When we used noise-free AFM images as training data, the model accurately extracted features and inferred structures from noise-free inputs. However, it lacked robustness to noisy inputs: inference accuracy degraded substantially when the reference images (inputs) were heavily contaminated by noise. By contrast, using images with comparatively high noise levels for training introduced a trade-off between peak accuracy and robustness. Specifically, even when the input noise matched the training noise, the prediction accuracy did not reach that of the noise-free case. Nevertheless, unlike the noise-free setting, the model retained reasonable accuracy even when the input noise level differed from that of the training data. 

%In general, the noise level of the training data should match that of the target images. Based on these results, the recommended practice depends on the noise level of the reference images. If the reference images have high noise levels ($\gtrsim 0.3$ nm), one should accept somewhat lower reliability, but reasonable accuracy can still be expected even when the input noise level differs from the training data. Conversely, if the reference images are low-noise, highly reliable predictions are achievable, but only when the input noise level is similar to that of the training data; images with markedly different noise levels should not be used as references.

\begin{figure}[htbp]
    \centering
    \includegraphics[width=\linewidth]{images/figures-03.png}
    \caption{
        \textbf{Correlation coefficient and robustness against noise}. 
        Panels (a) and (b) present the evaluations under noise-free conditions, whereas panels (c–f) evaluate the robustness of the estimations against noise.
        \textbf{(a)} Density of correlation coefficients (c.c.) between the given images (``experimental'' pseudo-AFM images) and pseudo-AFM images, each of which was generated by a rigidi-body fitting of a esimated conformation to the reference image maximizing teh c.c. The average of c.c. is indicated by the borken line (0.998). 
        \textbf{(b)} Scatter plot of the c.c. and the RMSD. 
        %indicating a negative correlation between image similarity and structural accuracy.
        \textbf{(c)} Scatter plot of the c.c. and the RMSD using ``experimental'' pseudo-AFM images contaminated by Gaussian noise with a standard deviation of 0.3 nm. 
        %Compared with panel (b), which was computed using noise-free images, the c.c. values are overall lower, and this is accompanied by a general upward shift in the RMSD distribution.
        \textbf{(d)} Examples of pseudo-AFM images at different noise levels used in the investigation of noise-robustness. 
        The images were generated from randomly selected structures from the MD data and then adding Gaussian noise with various standard deviations.
        \textbf{(e)} Heat map of mean RMSD between the estimated structures and the ground-truth structures 
        for various noise levels in the training and ``experimental'' pseudo-AFM data. 
        %under conditions where noise was introduced in both the training and reference data. 
        %The x axis indicates the noise level in the training data, and the y axis indicates that in the reference data. 
        Each mean RMSD was computed for 20 independent structure estimations. 
        \textbf{(f)} Heat map of mean c.c. between the rigid-body fitted pseudo-AFM images and the ``experimental'' pseudo-AFM images under the same noise levels with the panel (e). 
    }
    \label{fig:noise_robustness}
\end{figure}

\subsection{Application to real experimental AFM images: A Flagellar Protein \flhac}
\label{subsec:flhac_results}

FlhA is a key component of the flagellar protein export system, and its C-terminal domain, \flhac, helps transport proteins by assisting their binding to the exporter.
Structurally, it comprises four domains (\acdi, \acdii, \acdiii, and \acdiv) and a linker to the transmembrane domain (FlhA\textsubscript{TM}) (\Cref{fig:flhac_statistic}~(a)). 
Previous studies using X-ray crystallography and mutational analysis indicate that hinge motions among these domains influence transport activity and motility \cite{saijo-hamano2010flha}.

\paragraph{Training data preparation and trainning a g-CNN}
We selected \acdi--\acdiv and \acdii--\acdiii as inter-domain distance features.
The candidate conformations and their MolProbity scores are shown in \Cref{fig:flhac_selection}.

\begin{figure}[htbp]
    \centering
    \includegraphics[width=0.9\linewidth]{images/figures-04.png}
    \caption{
        \textbf{Results for \flhac}.
        \textbf{(a,d,g)} Reference AFM images selected from real experimental data. 
        \textbf{(b,e,h)} Crystal structures superposed onto the pseudo-AFM images generated from them.
        The pose of them were optimized to have the highest correlation values with the experimental AFM images through rigid-body fitting.
        \textbf{(c,f,i)} Estimated structures by \Model, superposed onto the pseudo-AFM images generated from them. 
    }
    \label{fig:flhac_results}
\end{figure}

\begin{figure}[htbp]
    \centering
    \includegraphics[width=0.9\linewidth]{images/figures-04.2.png}
    \caption{
        \textbf{Results for \flhac}.
        \textbf{(a)} Crystal structure of \flhac (PDB ID: 3A5I). 
        The four domains are colored as follows: dark blue, linker; sky blue, \acdi; green, \acdii; yellow, \acdiii; and orange, \acdiv. 
        \textbf{(b)} 
        Comparison of correlation coefficients (c.c.) between \Model predictions and the PDB database structure with the experimental AFM images. 
        The x axis represents the mean of $\mathrm{c.c._{AFM-Fold}}$ and $\mathrm{c.c._{PDB}}$, 
        and the x axis shows their difference ($\mathrm{c.c._{AFM-Fold}} - \mathrm{c.c._{PDB}}$). 
        Most data points lie above zero, indicating that \Model generally achieves higher agreement with the reference AFM images than the static PDB structure.
        \textbf{(c)} 
        Distribution of the inter-domain distance between \acdii and \acdiv. 
        The red histogram corresponds to the \Model's prediction (derived from 159~ms of HS-AFM observation).
        \textbf{(d)} 
        The transition of inter-domain distances estimated by \Model. 
        \textbf{(e)} 
        Time correlation of the inter-domain distance as a function of time lag. 
        The solid line shows the results from \Model predictions, while the dashed line corresponds to temporally shuffled predictions. 
    }
    \label{fig:flhac_statistic}
\end{figure}

\paragraph{Predictions from real AFM images}
To evaluate the applicability of \Model to real AFM data, we estimated the 3D structures using \Model from 159 HS-AFM snapshots of the \flhac monomer.
The randomly selected predictions are shown in \cref{fig:flhac_results}.

As a comparison, we perfoemed rigid-body fitting for both structures estimated by \Model and the PDB database structure.
For almost all cases, \Model reproduced more sililar images with reference images than the PDB database structure (see \cref{fig:flhac_statistic} (b)).

To evaluate the predicted conformations, we computed the inter-domain distance between \acdii and \acdiv for all estimated structures.
This distance serves as a useful indicator to distinguish between open and closed conformations.
The resulting histogram is shown in \cref{fig:flhac_statistic}(c).
For comparison, we also generated 100 structures using AlphaFlow\cite{jing2024alphaflow}, BioEmu\cite{bioemu2025}, and \AFii\cite{mirdita2022colabfold} with MSA subsampling, and plotted their distributions together.
Compared with these models, \Model tends to predict more open conformations.

To interpret this result, we analyzed the predicted conformations as a time series.
As shown in \cref{fig:flhac_statistic}(e), the time correlation between adjacent frames (with a time lag of 1~ms) remains relatively high ($ \sim 0.5$).
Moreover, \Model does not simply predict constant values; as illustrated in \cref{fig:flhac_statistic}(d), the predicted inter-domain distances exhibit substantial transitions over time.
When we randomly shuffled the frame order, the temporal correlation nearly vanished, as indicated by the dotted line in \cref{fig:flhac_statistic}(e).

Although \Model relies solely on AFM images, its ability to predict similar structures for neighboring frames suggests that it is robust to experimental noise and variations in probe shape.
This consistency indicates that \Model can extract meaningful structural information from real AFM data.


\section{Discussion and Conclusions}

In this work, we have proposed a computational protocol, \Model, to predict the 3D conformation of biomolecules from AFM images. 
\Model enables rapid and accurate conformation estimation compared to previous approaches.
\Model extracts features from AFM images with a group-invariant (symmetry-aware) CNN, reducing computational cost in two ways.
First, it reduces the amount of required training data, because symmetry-aware representations learn efficiently from limited images \cite{e2cnn}.
Second, it shortens inference time: guiding \AFiii requires computing $\ptc{\phi_{\text{object}}}{X_t}$; replacing this with $\mathcal{L}_{\text{MSE}}(X_t)$ keeps the computational cost per structure low.
In the estimations of this study, per-frame estimation finished within one minute on a single GPU.
It has the potential to serve as a powerful tool for interpreting HS-AFM movies comprised of hundreds of frames in both structural and dynamic aspects.

%\paragraph{Merits}
% Candidates: Advantages, Merits, Benefits

%Across our evaluations, \Model delivered competitive prediction accuracy at low computational cost.
%Therefore, \Model can help deepen understanding of biological processes---for example, by quantifying large-scale domain rearrangements along putative pathways and by prioritizing follow-up measurements in single-molecule studies.

As a limitation of \Model, it requires the CVs to be estimated from the AFM image, which is a pre-processing step.
The \Model framework can be readily extended by altering the physical observables it incorporates.
While we used inter-domain distances here, it can incorporate residue-level distance restraints, enabling AFM-guided reconstructions that capture finer-grained motions without a disproportionate increase in runtime.

%\paragraph{Limitations \& Future Work}
% Candidates: Weaknesses, Constraints, Challenges
\Model infers conformations from a single AFM image and currently does not quantify predictive uncertainty due to image noise or molecular orientation.
Extending the framework to estimate uncertainty would make the predictions easier to interpret \cite{Kendall2017Uncertainty}.
Moreover, AFM image sequences constitute time-series data in which poses change only slightly between adjacent frames; incorporating temporal dependencies---for example, with TimeSformer \cite{Bertasius2021TimeSformer} or non-local networks \cite{Wang2018NonLocalNetworks}---could further improve fidelity.
Finally, in this work we steered \AFiii's sampling using AFM images; consequently, neither thermodynamic stability nor explicit physical priors is incorporated.
In the future, conditioning the prior of an ensemble generative model should enable integrated modeling that unifies experimental measurements, physical priors, and computational inference.

\section{Code Availability Statement}
text here

\section{Data Availability Statement}
text here

\section{Acknowledgements}
\label{sec:acknowledgements}
We appreciate Professor Tohru Minamino (The University of Osaka) and Professor Noriyuki Kodera (Kanazawa University) for kindly providing the experimental AFM image data used in this work. This work was supported by MEXT as "Program for Promoting Researches on the Supercomputer Fugaku" (Development and application of large-scale simulation-based inferences for biomolecules JPMXP1020230119) and used computational resources of supercomputer Fugaku provided by the RIKEN Center for Computational Science (Project IDs: hp230209 hp240215, hp250233). 



\section{Conclusion}
\label{sec:conclusion}

In this work, we proposed a novel model, \Model, to predict the 3D conformation of biomolecules directly from AFM images. 
\Model enables accurate and fast conformation estimation compared to previous approaches. 
With further improvements, it has the potential to advance the understanding of biochemical reactions and serve as a powerful tool for exploring biological phenomena in both structural and dynamic aspects.

\section{Acknowledgements}
\label{sec:acknowledgements}
We thank Professor Kodera at Kanazawa University for kindly providing the experimental AFM data used in this work.


\printbibliography

\appendix

\setcounter{section}{0}
\renewcommand{\thesection}{S\arabic{section}}
\setcounter{equation}{0}
\renewcommand{\theequation}{S\arabic{equation}}
\setcounter{figure}{0}
\renewcommand{\thefigure}{S\arabic{figure}}
\setcounter{table}{0}
\renewcommand{\thetable}{S\arabic{table}}
\setcounter{algorithm}{0}
\renewcommand{\thealgorithm}{S\arabic{algorithm}}

\section{Navigating scheduling}
\label{app:guidance_sch}

We investigated how the navigating schedule $\eta_t$ in \Cref{eq:af3-guided} affects the navigating of \AFiii structure generation. 
The conditional generation process we used in this study is formulated as:

\begin{equation}
    X_{t_{i+1}} = X_{t_i} - \eta (t_{i+1} - \hat{t}_i) \bigl( \nabla_{X_{t_i}} \log p_{t_i}(X_{t_i} | a) - \eta_{t_i} \nabla_{X_{t_i}} \mathcal{L}_{\text{MSE}}(X_{t_i}) \bigr) + \lambda \sqrt{\hat{t}_i^2 - t_i^2} \epsilon,
    \label{eq:af3-guided-hatt}
\end{equation}
where $i$ propagates from $1$ to $N$ and $\eta_t$ dominates the strength of the navigation. 

Prior work commonly used simple schedules—--e.g., a constant value \cite{maddipatla2025inverse} or a mid-trajectory sigmoid \cite{raghu2025multiscale}—--
but, to our knowledge, there has not been a systematic quantification of how these choices affect generated structures.
Here, across different navigating schedules, we evaluated the trade-off 
between the accuracy of following the target CV values and the physical plausibility of the generated structures, 
and found an optimal navigating parameter set.

%\subsection{Experimental settings}
%\label{app:guidance_exp}

\paragraph{Navigating family and design of the sweep.}
We defined navigating schedules $\eta_t$ from the sigmoidal family defined in \Cref{eq:sigmoid}.
Specifically, we vary two hyperparameters that determine \emph{when} navigating turns on ($\tau_{\text{start}}$) and \emph{how strong} it becomes at the end ($y_{\text{max}}$).
For the two hyperparameters, we select five candidate values each: 
$\tau_{\text{start}}$ is chosen at evenly spaced points in $[0.4,0.9]$, 
and $y_{\text{max}}$ is chosen at evenly spaced points in $[0.6,1.6]$. 
By evaluating all combinations of these choices, we obtained $5\times 5 = 25$ distinct schedule settings in total. 
For each setting we generated five independent conformations and computed the averages of evaluation metrics of them. 

\begin{equation}
    \begin{split}
        \eta_{t_i} =
        \frac{y_{\text{max}}}{2}
        \frac{\| \delta \|_2}{\| g \|_2}
        \left(
            1 +
            \frac{\tanh\!\left( \alpha(\tau_i - \beta) / 2 \right)}
                 {\tanh\!\left( \alpha\beta / 2 \right)}
        \right),
        \qquad
        \tau_i = \frac{i / N - \tau_{\text{start}}}{1 - \tau_{\text{start}}},
        \\[1.5ex]
        \delta = \nabla_{X_{t_i}}\log p_{t_i}(X_{t_i}),
        \quad
        g = - \nabla_{X_{t_i}} \mathcal{L}_{\text{MSE}}(X_{t_i}),
        \quad
        \alpha = 10.0,
        \quad
        \beta = 0.5.
        \label{eq:sigmoid}
    \end{split}
\end{equation}

\paragraph{Systems and collective variables}
Here, we describe the choice of protein system and the collective variables (CVs) on which restraints are applied.  
The way restraints are imposed can in principle depend on factors such as protein size or the specific CVs chosen, 
but our goal is to provide a concrete reference point that can serve as a guideline when users consider how to set the guidance strength.  

In this experiments, we investigated the relationship between the guidance strength and the physical plausibility of the generated structures for both \AK (214 residues) and \flhac (346 residues).
The choice of CVs used for applying the guidance followed the same setup as in the main part---that is, LID–CORE, CORE–AMPbd, and AMPbd–LID for \AK, and \acdi–\acdiv and \acdii–\acdiii for \flhac.
As a reference, under guidance-free conditions, \AFiii predicts structures with inter-domain distances of $(2.85\,\mathrm{nm},\,2.19\,\mathrm{nm},\,3.45\,\mathrm{nm})$ for \AK, and $(3.47\,\mathrm{nm},\,4.38,\mathrm{nm})$ for \flhac.
We conducted two types of experiments for these proteins:
(i) \textbf{in-distribution targets}, which correspond to physically plausible inter-domain distances observed in MD simulations. Specifically, $(2.41\,\mathrm{nm},\,2.01,\mathrm{nm},\,2.68,\mathrm{nm})$ was chosen for \AK and $(2.5\,\mathrm{nm},\,4.5\,\mathrm{nm})$ for \flhac.
(ii) \textbf{out-of-distribution targets}, which represent physically implausible inter-domain distances not observed in MD simulations. In this case, $(1.21\,\mathrm{nm},\,1.51,\mathrm{nm},\,0.53,\mathrm{nm})$ was chosen for \AK and $(2.0\,\mathrm{nm},\,3.0,\mathrm{nm})$ for \flhac.



%\subsection{Results}
%\label{app:guidance_results}

Across both the in-distribution and out-of-distribution targets, three clear patterns emerge.  

\begin{itemize}
    \item First, there is a dividing line in the MSE landscape, approximately connecting $(\tau_{\text{start}}, y_{\text{max}})=(0.65,\,0.60)$ and $(0.73,\,1.60)$ (see  \Cref{fig:guidance_scheduling_ak} (b, h) and \Cref{fig:guidance_scheduling_flhac} (b, h)).
        On the right side of this line (i.e., with later $\tau_{\text{start}}$ and smaller $y_{\text{max}}$), the MSE between the target CVs and the generated conformation's CVs rises sharply. 
    \item Second, the MolProbity-score contours are roughly circular, centered near $(\tau_{\text{start}}, y_{\text{max}})=(0.40,\,1.60)$, rather than linear (see \Cref{fig:guidance_scheduling_ak} (a, g) and \Cref{fig:guidance_scheduling_flhac} (a, g)).
    \item Third, combining these two observations, we find that starting the guidance early (small $\tau_{\text{start}}$) and keeping the restraint weak (low $y_{\text{max}}$) enables the application of constraints 
        without substantially compromising the physical plausibility of the generated structures.
\end{itemize}

\begin{figure}[htbp]
    \centering
    \includegraphics[width=0.9\linewidth]{images/figures-05.png}
    \caption{
        All navigating schedulings evaluated in this study. 
        The setting used in our main experiments, $\tau_{\text{start}}=0.6$ and $y_{\text{max}}=0.6$, is highlighted with a thick red line.
        }
    \label{fig:guidance_scheduling}
\end{figure}

\begin{figure}[htbp]
    \centering
    \includegraphics[width=0.9\linewidth]{images/figures-06.png}
    \caption{
        \textbf{Results for \AK}. The upper 6 panels ((a)--(f)) shows results for the in-distribution target, and the lower 6 panels ((g)--(l)) shows results for the out-of-distribution target. 
        \textbf{(a), (g)} Average MolProbity scores. 
        \textbf{(b), (h)} Mean squared error (MSE). 
        \textbf{(c), (i)} The inter-domain distance of estimated conformations (25 points, each one represents average inter-domain distance of corresponding schedule setting), with point colors representing the MolProbity scores. 
        The cross point means that of the crystal structures (PDBID: 4AKE (open) and 1AKE (close)), and circle point means that of the target (every estimated structures were generated navigated towards this point).
        \textbf{(d), (e), (f), (j), (k), (l)} Evaluation metrics obtained during MolProbity scoring: clash score, Ramachandran favored rate (\%), and rotamer outlier rate (\%).
        }
    \label{fig:guidance_scheduling_ak}
\end{figure}

\begin{figure}[htbp]
    \centering
    \includegraphics[width=0.9\linewidth]{images/figures-07.png}
    \caption{
        \textbf{Results for \flhac}. As before, the upper 6 panels ((a)--(f)) shows results for the in-distribution target, and the lower 6 panels ((g)--(l)) shows results for the out-of-distribution target. 
        \textbf{(a), (g)} Average MolProbity scores. 
        \textbf{(b), (j)} Mean squared error (MSE). 
        \textbf{(c), (i)} The inter-domain distance of estimated conformations, with point colors representing the MolProbity scores. 
        The cross point means that of the crystal structure (PDBID: 3A5I), and circle point means that of the target.
        \textbf{(d), (e), (f), (j), (k), (l)} Evaluation metrics obtained during MolProbity scoring: clash score, Ramachandran favored rate (\%), and rotamer outlier rate (\%).
        }
    \label{fig:guidance_scheduling_flhac}
\end{figure}

%\subsection{Conclusion}
Overall, we observed the expected trade-off: stronger guidance improves agreement with the target, but tends to degrade structural plausibility. 
At the same time, we obtained an additional insight: rather than applying strong restraints only in the final stages of the reverse process, it is more effective to apply weak restraints from the early stages.  
In fact, regions with low MSE consistently appear in the area of $\hat{t}_{\text{start}} \lessapprox 0.6$ (see \Cref{fig:guidance_scheduling_ak} (b,h) and \Cref{fig:guidance_scheduling_flhac} (b,h)), 
and among them, regions with low MolProbity scores are located where $\hat{y}_{\text{max}}$ is relatively small (see \Cref{fig:guidance_scheduling_ak} (a,g) and \Cref{fig:guidance_scheduling_flhac} (a,g)).

Based on these observations, in our study we adopt $\hat{t}_{\text{start}}=0.6$ and $y_{\text{max}}=0.6$ that allow effective guidance without substantially degrading structural quality.

\section{Model Implementation Details}
\label{app:additional_alg}

\begin{algorithm}[H]
\caption{\textsc{Forward function of $(\R^2, +) \rtimes C_8$ invariant CNN}}
\label{alg:cnscnn}
\begin{algorithmic}
\State \textbf{Input:} $I \in \mathbb{R}^{B \times D \cdot N \times H \times W}$, $D=1$, $N=8$, $H=W=35$; \\
channels = [3,6,6,12,12,8], kernel\_size = [7,5,5,5,5,5], \\
$N_{\text{hidden-layers}} = 3$, $D_{\text{hidden}} = 64$, $D_{\text{out}} = N_{\text{domain-pairs}}$
\State \textbf{Output:} $\mathbf{D} \in \mathbb{R}^{B \times D_{\text{out}}}$

\State $I \gets \mathrm{R2Conv}(I;\, \text{in}=D \cdot N,\, \text{out}=\mathrm{channels}[0] \cdot N,\, k=\mathrm{kernel\_size}[0]) \to \mathrm{InnerBatchNorm} \to \mathrm{ReLU}$
\For{$i = 1 \to \mathrm{len(channels)}-1$}
    \State $I \gets \mathrm{R2Conv}(I;\, \text{in}=\mathrm{channels}[i-1]\!\cdot N,\, \text{out}=\mathrm{channels}[i]\!\cdot N,\, k=\mathrm{kernel\_size}[i])$
    \State $I \gets \mathrm{InnerBatchNorm}(I) \to \mathrm{ReLU}(I)$
    \If{$i \bmod 2 = 0$}
        \State $I \gets \mathrm{PointwiseAvgPoolAntialiased}\!\left(I;\, \sigma=0.66,\, \mathrm{stride}=
        \begin{cases}
        2 & \text{if size odd}\\
        1 & \text{if size even}
        \end{cases}\right)$
    \EndIf
\EndFor

\State $I' \gets \mathrm{GroupPooling}(I)$ \hfill $I' \in \mathbb{R}^{B \times \mathrm{channels}[-1] \times H' \times W'}$
\State $I'' \gets \mathrm{PointwiseAvgPool}(I')$ \hfill $I'' \in \mathbb{R}^{B \times \mathrm{channels}[-1] \times 1 \times 1}$
\State $\mathbf{z} \gets \mathrm{Flatten}(I'')$ \hfill $\mathbf{z} \in \mathbb{R}^{B \times \mathrm{channels}[-1]}$
\State $\mathbf{h} \gets \mathrm{Linear}(\mathbf{z};\, \text{in}=\mathrm{channels}[-1],\, \text{out}=D_{\text{hidden}}) \to \mathrm{BatchNorm1d} \to \mathrm{ELU}$ \hfill $\mathbf{h} \in \mathbb{R}^{B \times D_{\text{hidden}}}$
\For{$j = 1 \to N_{\text{hidden-layers}}-1$}
    \State $\mathbf{h} \gets \mathrm{Linear}(\mathbf{h};\, \text{in}=D_{\text{hidden}},\, \text{out}=D_{\text{hidden}}) \to \mathrm{ReLU}(\mathbf{h})$ \hfill $\mathbf{h} \in \mathbb{R}^{B \times D_{\text{hidden}}}$
\EndFor
\State $\mathbf{D} \gets \mathrm{Linear}(\mathbf{h};\, \text{in}=D_{\text{hidden}},\, \text{out}=D_{\text{out}})$ \hfill $\mathbf{D} \in \mathbb{R}^{B \times D_{\text{out}}}$
\State \Return $\mathbf{D}$
\end{algorithmic}
\end{algorithm}

%\section{Supporting figures}
%\label{app:additional_fig}

\begin{figure}[H]
    \centering
    \includegraphics[width=\linewidth]{images/figures-08.png}
    \caption{
        \textbf{Selected candidate conformations for \AK and MolProbity evaluation for them}.
        \textbf{(a)} Projection of the candidate conformations into the inter-domain distance space. 
        The following panels shows
        \textbf{(b)} MolProbity score,
        \textbf{(c)} Clash score,
        \textbf{(d)} Ramachandran favored rate (\%), and
        \textbf{(e)} Rotamer outlier rate (\%) of the candidate conformations.
        }
    \label{fig:ak_selection}
\end{figure}

\begin{figure}[H]
    \centering
    \includegraphics[width=0.8\linewidth]{images/figures-09.png}
    \caption{
        \textbf{Selected candidate conformations for \flhac and MolProbity evaluation for them}.
        \textbf{(a)} Projection of the candidate conformations into the inter-domain distance space. 
        The following panels shows
        \textbf{(b)} MolProbity score, 
        \textbf{(c)} Clash score,
        \textbf{(d)} Ramachandran favored rate (\%), and
        \textbf{(e)} Rotamer outlier rate (\%) of the candidate conformations.
        }
    \label{fig:flhac_selection}
\end{figure}


\end{document}